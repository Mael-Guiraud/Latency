%!TEX root = Manuscript.tex

\usepackage{amsmath,amssymb,amsthm,amsfonts}
\usepackage[T1]{fontenc}
\usepackage[utf8]{inputenc}
\usepackage[english]{babel}

\usepackage[left=2.0cm,right=2.0in,top=2.0in,bottom=2.0in,includefoot,includehead,headheight=13.6pt]{geometry}
\renewcommand{\baselinestretch}{1.5}

\usepackage{silence}

\WarningFilter{minitoc(hints)}{W0023}
\WarningFilter{minitoc(hints)}{W0024}
\WarningFilter{minitoc(hints)}{W0028}
\WarningFilter{minitoc(hints)}{W0030}

\usepackage{aecompl}
\usepackage{url}

% List of abbreviations
% Do not try putting acronyms in section titles, that would cause infinite loop of pdftex compilation
\usepackage[printonlyused,withpage]{acronym}

% My pdf code

\usepackage{ifpdf}

\ifpdf
  \usepackage[pdftex]{graphicx}
\else
  \usepackage{graphicx}
\fi

\graphicspath{{.}{images/}}

% Links in pdf
\usepackage{color}
\definecolor{linkcol}{rgb}{0,0,0.4}
\definecolor{citecol}{rgb}{0.5,0,0}

% Table of contents for each chapter

\usepackage[nottoc, notlof, notlot]{tocbibind}
\usepackage{minitoc}
\setcounter{minitocdepth}{2}
\mtcindent=15pt
% Use \minitoc where to put a table of contents

% definitions.
% -------------------

\setcounter{secnumdepth}{3}
\setcounter{tocdepth}{2}

% Some useful commands and shortcut for maths:  partial derivative and stuff

\newcommand{\pd}[2]{\frac{\partial #1}{\partial #2}}
\def\abs{\operatorname{abs}}
\def\argmax{\operatornamewithlimits{arg\,max}}
\def\argmin{\operatornamewithlimits{arg\,min}}
\def\diag{\operatorname{Diag}}
\newcommand{\eqRef}[1]{(\ref{#1})}

\usepackage{rotating}                    % Sideways of figures & tables
%\usepackage{bibunits}
%\usepackage[sectionbib]{chapterbib}          % Cross-reference package (Natural BiB)
%\usepackage{natbib}                  % Put References at the end of each chapter
                                         % Do not put 'sectionbib' option here.
                                         % Sectionbib option in 'natbib' will do.
\usepackage{fancyhdr}                    % Fancy Header and Footer

% \usepackage{txfonts}                     % Public Times New Roman text & math font

%%% Fancy Header %%%%%%%%%%%%%%%%%%%%%%%%%%%%%%%%%%%%%%%%%%%%%%%%%%%%%%%%%%%%%%%%%%
% Fancy Header Style Options

\pagestyle{fancy}                       % Sets fancy header and footer
\fancyfoot{}                            % Delete current footer settings

%\renewcommand{\chaptermark}[1]{         % Lower Case Chapter marker style
%  \markboth{\chaptername\ \thechapter.\ #1}}{}} %

%\renewcommand{\sectionmark}[1]{         % Lower case Section marker style
%  \markright{\thesection.\ #1}}         %

\fancyhead[LE,RO]{\bfseries\thepage}    % Page number (boldface) in left on even
% pages and right on odd pages
\fancyhead[RE]{\bfseries\nouppercase{\leftmark}}      % Chapter in the right on even pages
\fancyhead[LO]{\bfseries\nouppercase{\rightmark}}     % Section in the left on odd pages

\let\headruleORIG\headrule
\renewcommand{\headrule}{\color{black} \headruleORIG}
\renewcommand{\headrulewidth}{1.0pt}
\usepackage{colortbl}
\arrayrulecolor{black}

\fancypagestyle{plain}{
  \fancyhead{}
  \fancyfoot{}
  \renewcommand{\headrulewidth}{0pt}
}

\usepackage{algorithm}
\usepackage[noend]{algorithmic}


\usepackage{scrextend}

%%% Clear Header %%%%%%%%%%%%%%%%%%%%%%%%%%%%%%%%%%%%%%%%%%%%%%%%%%%%%%%%%%%%%%%%%%
% Clear Header Style on the Last Empty Odd pages
\makeatletter

\def\cleardoublepage{\clearpage\if@twoside \ifodd\c@page\else%
  \hbox{}%
  \thispagestyle{empty}%              % Empty header styles
  \newpage%
  \if@twocolumn\hbox{}\newpage\fi\fi\fi}

\makeatother

%%%%%%%%%%%%%%%%%%%%%%%%%%%%%%%%%%%%%%%%%%%%%%%%%%%%%%%%%%%%%%%%%%%%%%%%%%%%%%%
% Prints your review date and 'Draft Version' (From Josullvn, CS, CMU)
\newcommand{\reviewtimetoday}[2]{\special{!userdict begin
    /bop-hook{gsave 20 710 translate 45 rotate 0.8 setgray
      /Times-Roman findfont 12 scalefont setfont 0 0   moveto (#1) show
      0 -12 moveto (#2) show grestore}def end}}
% You can turn on or off this option.
% \reviewtimetoday{\today}{Draft Version}
%%%%%%%%%%%%%%%%%%%%%%%%%%%%%%%%%%%%%%%%%%%%%%%%%%%%%%%%%%%%%%%%%%%%%%%%%%%%%%%

\newenvironment{maxime}[1]
{
\vspace*{0cm}
\hfill
\begin{minipage}{0.5\textwidth}%
%\rule[0.5ex]{\textwidth}{0.1mm}\\%
\hrulefill $\:$ {\bf #1}\\
%\vspace*{-0.25cm}
\it
}%
{%

\hrulefill
\vspace*{0.5cm}%
\end{minipage}
}

\let\minitocORIG\minitoc
\renewcommand{\minitoc}{\minitocORIG \vspace{1.5em}}

\usepackage{subfigure}
\usepackage{multirow}
% \usepackage{slashbox}

\newenvironment{bulletList}%
{ \begin{list}%
	{$\bullet$}%
	{\setlength{\labelwidth}{25pt}%
	 \setlength{\leftmargin}{30pt}%
	 \setlength{\itemsep}{\parsep}}}%
{ \end{list} }

\newtheorem{definition}{Definition}
\renewcommand{\epsilon}{\varepsilon}

% centered page environment

\newenvironment{vcenterpage}
{\newpage\vspace*{\fill}\thispagestyle{empty}\renewcommand{\headrulewidth}{0pt}}
{\vspace*{\fill}}

% Hyperref code

\ifpdf
  \usepackage[hyperindex=true]{hyperref}
\else
  \usepackage[dvipdfm,hyperindex=true]{hyperref}
\fi


%\renewcommand*{\backrefsep}{, }
%\renewcommand*{\backreftwosep}{ and~}
%\renewcommand*{\backreflastsep}{ and~}

% Change this to change the informations included in the pdf file

% See hyperref documentation for information on those parameters

\hypersetup
{
bookmarksopen=true,
pdftitle="Manuscript title",
pdfauthor="Your name",
pdfsubject="Manuscript topic in a few words", %subject of the document
%pdftoolbar=false, % toolbar hidden
pdfmenubar=true, %menubar shown
pdfhighlight=/O, %effect of clicking on a link
colorlinks=true, %couleurs sur les liens hypertextes
pdfpagemode=UseNone, %aucun mode de page
pdfpagelayout=SinglePage, %ouverture en simple page
pdffitwindow=true, %pages ouvertes entierement dans toute la fenetre
linkcolor=linkcol, %couleur des liens hypertextes internes
citecolor=citecol, %couleur des liens pour les citations
urlcolor=linkcol %couleur des liens pour les url
}

\newcommand{\todo}[1]{{\color{red} TODO: {#1}}}

\usepackage{complexity}
\usepackage{setspace}
\usepackage{xspace}
\usepackage{caption}
\usepackage{longtable}
\usepackage{tkz-graph}
\usepackage{float}
\usepackage{tabularx}
\usepackage{icomma}
\newcommand\shortestlongest{\texttt{ShortestLongest}\xspace}
\newcommand\metaoffset{\texttt{MetaOffset}\xspace}
\newcommand\ESCA{\texttt{ESCA}\xspace}
\newcommand\greedydeadline{\texttt{Greedy Deadline}\xspace}
\newcommand\MLS{\texttt{MLS}\xspace}
\newcommand\PMLS{\texttt{PMLS}\xspace}
\newcommand\ASPMLS{\texttt{ASPMLS}\xspace}

\newcommand\SPMLS{\texttt{SPMLS}\xspace}
\newcommand\FIFO{\texttt{FIFO}\xspace}
\newcommand\framepre{\texttt{FramePreemption}\xspace}
\newcommand\critdead{\texttt{CriticalDeadline}\xspace}

\newtheorem{proposition}{Proposition}
\newtheorem{theorem}{Theorem}

\newtheorem{fact}{Fact}
\newtheorem{lemma}[theorem]{Lemma}
\newtheorem{corollary}{Corollary}

% \renewcommand{\thefootnote}{\*}

\newcommand\pazl{\textsc{pazl}\xspace}
\newcommand\pall{\textsc{pall}\xspace}
\newcommand\wta{\textsc{wta}\xspace}
\newcommand\pra{\textsc{pra}\xspace}
\newcommand\minpazl{\textsc{minpazl}\xspace}
\newcommand\mintra{\textsc{mintra}\xspace}
\newcommand\minstra{\textsc{minstra}\xspace}
\newcommand{\spall}{\textsc{SPALL}\xspace}
\newcommand{\bra}{\textsc{BRA}\xspace}
\newcommand{\ADO}{\textsc{ADO}\xspace}
\newcommand\greedynormalized{\texttt{Greedy Normalized}\xspace}
\newcommand\greedypacked{\texttt{Greedy Packed}\xspace}
\newcommand\hybridgreedydeadline{\texttt{Hybrid Greedy Deadline}\xspace}
\newcommand\hybridgreedynormalized{\texttt{Hybrid Greedy Normalized}\xspace}
\newcommand\hgn{\texttt{HGN}\xspace}
\usepackage{diagbox}

\newcommand\pma{\textsc{pazl}\xspace}
\newcommand\firstfit{\texttt{First Fit}\xspace}
\newcommand\compactpair{\texttt{Compact Pairs}\xspace}
\newcommand\greedyuniform{\texttt{Greedy Uniform}\xspace}
\newcommand\swapandmove{\texttt{Swap and Move}\xspace}
\newcommand\compactfit{\texttt{Compact Fit}\xspace}
\newcommand\greedypotential{\texttt{Greedy Potential}\xspace}
\newcommand\exactresolution{\texttt{Exhaustive Search of Compact Assignments}\xspace}



\newcommand\notationdelay{\delta}
\newcommand\nomdelay{delay}
\newcommand\nomdelaypluriel{delays }

\usepackage{comment} 
\usepackage{amsmath}
\usepackage{amsfonts}
\usepackage{fancyhdr}
\usepackage{amssymb}
\usepackage{color} % où xcolor selon l'installation
\definecolor{Prune}{RGB}{99,0,60}
\usepackage{mdframed}
\usepackage{multirow} %% Pour mettre un texte sur plusieurs rangées
\usepackage{multicol} %% Pour mettre un texte sur plusieurs colonnes
\usepackage{scrextend} %Forcer la 4eme  de couverture en page pair
\usepackage{tikz}
\usepackage{graphicx}
\usepackage[absolute]{textpos} 
\usepackage{colortbl}
\usepackage{array}
\usepackage[backend=bibtex]{biblatex}
\addbibresource{Src.bib}
\usetikzlibrary{arrows,decorations.markings}