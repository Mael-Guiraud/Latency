%!TEX root = Manuscript.tex
%%%%%%%%%%%%%%%%%%%%%%%%%%%%%%%%%%%%%%%%%%%%%%%%%%%%%%%%%%%%%%%
% 4eme de couverture
\ifthispageodd{\newpage\thispagestyle{empty}\null\newpage}{}
\thispagestyle{empty}
\newgeometry{top=1.5cm, bottom=1.25cm, left=2cm, right=2cm}
\fontfamily{rm}\selectfont

\lhead{}
\rhead{}
\rfoot{}
\cfoot{}
\lfoot{}

\noindent 
%*****************************************************
%***** LOGO DE L'ED À CHANGER ÉVENTUELLEMENT *********
%*****************************************************
\includegraphics[height=2.45cm]{index}
\vspace{1cm}
%*****************************************************

\begin{mdframed}[linecolor=Prune,linewidth=1]
\vspace{-.25cm}
\paragraph*{Titre:} Ordonnancement periodiques de messages pour minimiser la latence dans les réseaux dans un contexte 5G et au delà


\begin{small}
\vspace{-.25cm}
\paragraph*{Mots clés:}  Cloud Radio Access Network, Ordonnancement periodique, Heuristiques de recherche locale, Analyse de complexité, Réduction de la latence, Theorie des graphes

\vspace{-.5cm}
\begin{multicols}{2}
\paragraph*{Résumé:} Cette thèse est le fruit d’une collaboration entre les laboratoires DAVID et Nokia Bell Labs France.
L’idée originale est de trouver des solutions algorithmiques pour gérer des flux periodiques de manière déterministe dans les réseaux afin de contrôler et de minimiser le temps de transmission, appelé latence. L’un des objectifs de la 5G (le C-RAN, pour Cloud Radio Access Network) est de centraliser les unités de calculs des antennes radio des réseaux de télécommunications (appelé Radio Access Network) dans un même centre de calcul (le Cloud). Le réseau entre le centre de calcul et les antennes doit être capable de satisfaire les contraintes de latence imposées par les protocoles.

Nous définissions le problème de trouver un ordonnancement periodique pour les messages de façon à ce qu'ils ne se disputent jamais la même ressource, et prouvons que les différentes variantes du problème étudiés sont NP-complets. Nous étudions dans un premier temps le problème pour une topologie particulière dans laquelle tous les flux partagent un même lien. Nous proposons dans un premier temps des algorithmes polynomiaux, de plus en plus évolués, ainsi que des algorithmes FPT permettant de trouver une solution quand le nombre de route est raisonnable, ce qui est le cas des réseaux C-RAN.

Les algorithmes développés dans cette première partie n’étant pas applicables directement aux topologies plus générales, nous proposons ensuite une forme compacte au problème qui nous permet de définir une notion de voisinage efficace pour des heuristiques de recherches locales (descente, recherche tabou, recuit simulé). Nous utilisons cette forme compacte pour définir un algorithme Branch and Bound efficace quand le nombre de routes est modéré.
Nous proposons aussi une évaluation de performance des solutions proposés par rapport aux solutions courantes de gestion des flux et montrons que notre modèle est réalisable en pratique grâce aux nouveaux équipements en cours de développement.
\end{multicols}
\end{small}
\end{mdframed}

%************************************
\vspace{7cm} % ALIGNER EN BAS DE PAGE
%************************************
\fontfamily{fvs}\fontseries{m}\selectfont
\begin{tabular}{p{14cm}r}
\multirow{3}{16cm}[+0mm]{{\color{Prune} Maison du doctorat de l'Université Paris-Saclay, 2ème étage aile ouest, Ecole normale supérieure Paris-Saclay, 4 avenue des Sciences,91190 Gif sur Yvette, France}} & \\
\end{tabular}
\newpage
\fontfamily{rm}\selectfont
\noindent 

%*****************************************************
%***** LOGO DE L'ED À CHANGER ÉVENTUELLEMENT *********
%*****************************************************
\includegraphics[height=2.45cm]{index}
\vspace{1cm}
%*****************************************************
\begin{mdframed}[linecolor=Prune,linewidth=1]
\vspace{-.25cm}
\paragraph*{Title:} Deterministic scheduling of periodic datagrams for low latency in 5G and beyond 

\begin{small}
\vspace{-.25cm}
\paragraph*{Keywords:}  Cloud Radio Access Network, Periodic scheduling, Local search heuristics, complexity analysis, Latency reduction,  Graph theory
\vspace{-.5cm}
\begin{multicols}{2}
\paragraph*{Abstract:} This thesis is the result of a collaboration between DAVID Laboratory and Nokia Bell Labs France.
The original idea is to find algorithmic solutions to deterministically manage periodic flows in networks in order to control and minimize the transmission time, called latency. One of the objectives of 5G (C-RAN, for Cloud Radio Access Network) is to centralize the calculation units of the radio antennas of telecommunications networks (called Radio Access Network) in the same computer center (the Cloud). The network between the computing center and the antennas must be able to satisfy the latency constraints imposed by the protocols.

We define the problem of finding a periodic scheduling for messages so that they never compete for the same resource, and prove that the different variants of the problem studied are NP-complete. We first study the problem for a particular topology in which all the streams share the same link. We first propose polynomial algorithms of increased sophistication, and FPT algorithms that allow us to find a solution when the number of routes is reasonable, which is the case for C-RAN networks.

Since the algorithms developed in this first part are not directly adaptable to more general topologies, we then propose a canonical form to the problem which allows us to define an efficient neighborhood notion for local search heuristics (hill climbing, tabu search, simulated annealing). We use this canonical form to define an efficient Branch and Bound algorithm when the number of routes is moderate.
We also propose a performance evaluation of the proposed solutions compared to current flow management solutions, and show that our model is feasible in practice thanks to new equipment under development.

\end{multicols}
\end{small}
\end{mdframed}
%************************************
\vspace{9cm} % ALIGNER EN BAS DE PAGE
%************************************
\fontfamily{fvs}\fontseries{m}\selectfont
\begin{tabular}{p{14cm}r}
\multirow{3}{16cm}[+0mm]{{\color{Prune} Maison du doctorat de l'Université Paris-Saclay, 2ème étage aile ouest, Ecole normale supérieure Paris-Saclay, 4 avenue des Sciences,91190 Gif sur Yvette, France}} 
\end{tabular}