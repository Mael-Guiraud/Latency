%!TEX root = Manuscript.tex

\silentchapter{Présentation de la thèse}
\label{chap:introfr}
%\addcontentsline{toc}{chapter}{Présentation de la thèse}


Les travaux présentés dans cette thèse s'inscrivent dans le contexte du développement de la 5G, et sont plus particulièrement axés sur la réduction de la latence dans les réseaux cœurs des opérateurs.
L'un des objectifs pour la 5G est de garantir une latence bout en bout la plus faible possible.
Réduire la latence dans les réseaux permet non seulement d'améliorer la qualité de service des utilisateurs, mais ouvre également la porte au développement d'applications pour lesquelles le temps de réponse est critique (véhicules autonomes, industrie 4.0, \ldots).
Le cas d'application que nous étudions est le Cloud Radio Access Network abrégé en C-RAN. Le but du C-RAN est de centraliser les unités de calcul situées aux pieds de chaque antenne dans un ou plusieurs centres de calcul communs, afin de faciliter la maintenance et de réduire les coûts d'exploitation. Les antennes envoient périodiquement des messages aux centres de calcul, qui calculent une réponse et l'envoient aux antennes avec la même périodicité. Le temps écoulé entre l'envoi d'un message par une antenne et la réception de sa réponse doit être inférieur à une durée imposée par le protocole de communication radio. Ces messages envoyés sont très lourds et utilisent donc beaucoup de bande passante dans les réseaux.

La méthode actuelle de gestion des messages dans les réseaux, le multiplexage statistique, consiste à dimensionner chaque lien de façon à ce que le flux moyen de messages utilisant un lien puisse emprunter ce lien sans contrainte. Il arrive fréquemment que des flux envoient beaucoup de paquets d'un coup dans le réseau. Quand trop de paquets doivent utiliser un lien en même temps, on parle de contention. Les messages qui ne peuvent pas utiliser le lien directement sont mis dans une file d'attente, que nous appelons buffer de contention. Faire attendre les messages dans ces buffers de contention augmente la latence des paquets. Plus un réseau est chargé, plus il est probable d'avoir de hautes latences dues à la contention. Dans les réseaux C-RAN, les sources envoient périodiquement une grande quantité de messages nécessitant une garantie de faible latence, ils ne peuvent donc pas être gérés grâce au multiplexage statistique. C'est pourquoi nous proposons des solutions de gestion déterministe et périodique des flux : le calcul d'un ordonnancement qui définit les dates de passage de chaque paquet dans chaque nœud du réseau. Les sources envoient périodiquement des paquets dans le réseau, toutes selon la même période, fixée par le protocole. Les ordonnancements que nous calculons garantissent l'absence de collision des paquets, quand l'ordonnancement est répété à l'infini de manière périodique.

Plusieurs groupes de travail proposent aujourd'hui des solutions techniques (présentées dans le chapitre~\ref{chap:TSN}) pour aider à contrôler la latence dans les réseaux. Avec ces solutions, les équipements du réseau sont capables d'allouer les ressources de transmission pour certains flux à un instant donné. Des travaux au sein de Nokia Bell Labs visent à aller plus loin pour pouvoir réserver une partie des ressources à un temps donné pour un paquet donné. Il faut toutefois calculer les dates auxquelles les paquets doivent arriver dans les nœuds du réseau. Cette thèse se concentre sur le fait d'organiser les paquets, de façon à ce qu'ils n'entrent en collision dans aucun des nœuds, afin de supprimer les buffers de contention. Se passer complètement des buffers de contention n'est pas toujours possible, notamment lorsque les réseaux sont composés de beaucoup de nœuds. Dans ce cas, l'objectif de nos travaux est de minimiser le temps passé par les paquets dans les buffers de contention. Il est important de souligner que dans ce cas-là, le temps d'attente dans les buffers de contentions ne sont plus subis comme pour le multiplexage statistique, mais contrôlés et prévisibles.

Nous modélisons un réseau par un multigraphe orienté acyclique pondéré dont les sommets représentent les points de contention entre messages dans le réseau. Les poids des arcs représentent le temps physique de transmission entre deux points de contention. Deux messages rentrent en conflit s’ils doivent passer par le même point de contention au même moment. Nous considérons que le routage est donné, et nous cherchons à organiser les messages de façon à ce qu'il n'y ait pas de conflit dans le réseau. Nous étudions dans un premier temps le problème sur des réseaux simples et courants, constitués de deux points de contention en série. 
Nous définissons le problème de décision consistant à choisir la date de passage de chaque message dans chacun de ces deux points de contention de façon à ce qu'aucun message n'ait de conflit avec un autre dans le réseau. Ce problème ressemble à des problèmes classiques d'ordonnancement, mais l'envoi périodique de nos flux en fait un problème original et difficile. Nous prouvons dans le chapitre~\ref{chap:model} que le problème est $\NP$-complet, même sur des graphes orientés acycliques de faible degré ou de faible profondeur, par réduction de problèmes de coloration d'arcs ou de sommets. Nous proposons donc des heuristiques (algorithmes gloutons, métaheuristiques) qui nous permettent de trouver de bonnes solutions en temps polynomial pour tout type d'instanceet des algorithmes FPT (de complexité exponentielle en le nombre de routes, mais pas en les autres paramètres du problème) qui trouvent une solution optimale au problème et qui sont suffisamment rapides sur les instances simples que nous étudions.


Nous étudions dans le chapitre~\ref{chap:PAZL} le problème de l'organisation de flux non-synchronisés dans un réseau sans aucun buffer.
Même si pour l'instant les protocoles liés au C-RAN ne permettent pas de désynchroniser les antennes (ce qui pourrait être le cas pour de prochaines générations de réseaux mobiles), cette approche est applicable dans d'autres contextes, comme une usine où des robots ne nécessitant pas de synchronisation doivent communiquer rapidement avec un centre de contrôle.
Nous cherchons à calculer un temps de départ des messages au début de leur route de façon à ce qu'ils ne soient pas en conflit avec les autres messages, sans que ce temps de départ ne soit considéré comme du temps de contention.
Les solutions de ce problème sont toutes optimales en matière de latence : la latence des messages est égale au temps physique de transmission, car aucune latence n'est ajoutée aux messages à cause de la contention. Nous décrivons des algorithmes gloutons de plus en plus évolués visant à optimiser l'impact d'ajouter un message à la solution partielle calculée. Ces algorithmes nous permettent de garantir qu'une solution au problème existe quand la charge du réseau est inférieure à $40\%$ (et même jusqu'à $61\%$ pour des messages de taille $1$). Nous proposons aussi un algorithme FPT (quand le problème est paramétré par le nombre de routes) qui nous permet de calculer la solution optimale en un temps raisonnable quand le nombre de routes est inférieur à $20$. Nos résultats montrent que le problème ne peut pas être résolu sans buffer de contention quand la charge du réseau est supérieure à $80\%$.
C'est pourquoi nous traitons dans le chapitre~\ref{chap:PALL} le problème d'organiser les flux avec un buffer sur la route, de façon à offrir un plus grand degré de liberté. Nous étudions plus particulièrement le problème de minimisation, c'est-à-dire, trouver une solution qui minimise la latence maximale des routes. Nous proposons une approche en deux parties. Premièrement, nous choisissons les temps d'envoi des messages pour qu'il n'y ait pas de conflits sur le premier point de contention, et nous résolvons dans un second temps le problème de choisir le temps d'attente de chaque message dans le second point de contention. Pour cela, nous décrivons un algorithme polynomial basé sur le problème d'ordonnancement classique de la littérature, adapté à notre cadre périodique. Nous proposons aussi un algorithme FPT basé sur le même principe, mais qui garantit de trouver la solution optimale. Nous montrons que nous sommes capables de trouver des solutions pour lesquelles la latence est minimale pour $99.9\%$ des instances dans des réseaux très chargés, et que nos méthodes donnent des résultats excellents comparées au multiplexage statistique.

Dans le chapitre~\ref{chap:SPALL}, nous étudions le problème d'organiser des flux synchronisés sur tout type de DAG. Dans ce cas, tous les messages sont envoyés en même temps par les sources et nous nous permettons de faire attendre les messages dans des buffers à chaque point de contention du réseau. Nous étudions le problème de minimiser la plus grande latence dans le réseau. Nous commençons par décrire des algorithmes gloutons qui trouvent une solution réalisable pour n'importe quelle charge, qui servent de point de départ aux algorithmes de recherche locale utilisés ensuite. Nous introduisons une forme compacte du problème qui nous permet de définir une notion de voisinage entre les solutions afin d'explorer l'ensemble de ces dernières. Nous étudions les performances des algorithmes de recherche d'optimum local (hill-climbing, recherche tabou, recuit simulé) et nous proposons un algorithme Branch and Bound qui énumère l'ensemble des solutions sous forme compacte, en faisant suffisamment de coupes pour trouver la solution optimale rapidement. Nous montrons expérimentalement que l'algorithme Branch and Bound est capable de trouver une solution optimale en un temps raisonnable pour $12$ routes, tandis que le recuit simulé permet de trouver des solutions bien meilleures que le multiplexage statistique pour n'importe quelle taille d'instance.

Nous étudions ensuite, dans le chapitre~\ref{chap:BE}, l'impact de nos algorithmes d'ordonnancement, lorsque les flux C-RAN périodiques et prioritaires partagent le réseau avec des flux Best-Effort, non prioritaires et dont les arrivées suivent un processus stochastique. Nous proposons une méthode d'adaptation de nos algorithmes qui permet de lisser la charge des flux C-RAN tout au long de la période, sans augmenter la latence. Nos expériences montrent que, même si organiser les flux de façon déterministe comme nous le faisons requiert d'utiliser un peu plus de bande passante pour réserver les ressources, la latence moyenne des flux Best-Effort est meilleure qu'avec le multiplexage statistique. Nous montrons aussi le même genre de résultats dans un anneau optique où l'ordonnancement des flux C-RAN est rendu trivial par les contraintes techniques de la conversion opto-électronique.

Toutes nos approches se basent sur des hypothèses techniques fortes : les flux doivent être parfaitement synchronisés, le réseau doit être intelligent et programmable. Le chapitre~\ref{chap:TSN} fait le point sur les standards récemment développés qui se rapprochent de nos hypothèses. Nous montrons aussi les limites de ces standards, et nous introduisons un équipement en phase de développement qui nous permettrait de réduire la latence dans les réseaux au temps physique de transmission.



\silentchapter{Introduction}
\label{chap:introen}

The work presented in this thesis takes place in the context of the development of 5G, and is more particularly focused on the reduction of latency in operators' core networks.
One of the objectives for 5G is to ensure the lowest possible end-to-end latency.
Reducing latency in networks not only improves the quality of service for users, but also opens the door to the development of applications for which response time is critical (autonomous vehicles, Industry 4.0, \ldots).
The application case we are studying is the Cloud Radio Access Network abbreviated to C-RAN. The goal of C-RAN is to centralize the computing units located at the feet of each antenna in one or more common data-centers, in order to facilitate maintenance and reduce operating costs. The antennas periodically send messages to the data-centers, which compute an answer and send it to the antennas with the same periodicity. The time elapsed between the sending of a message by an antenna and the reception of its answer must be less than a deadline imposed by the radio communication protocol. These messages are very heavy and therefore use a lot of bandwidth in networks.

The current way to manage messages in a network is called statistical multiplexing. It consists of dimensioning each link so that the average flow of messages using a link can use this link without constraint. It is not uncommon for flows to send many packets at once through the network. When too many packets have to use a link at the same time, this is called contention. Messages that cannot use the link directly are put into a buffer, which we call a contention buffer. Buffering messages in these contention buffers increases packet latency. The more messages there is in a network, the highest are latencies due to contention. In C-RAN networks, sources periodically send large amounts of messages that require a guarantee of low latency, so they cannot be handled by statistical multiplexing. This is why we offer solutions for deterministic and periodic flow management: we compute a scheduling that defines the dates on which each packet passes through each node of the network. The sources periodically send packets in the network, all according to the same period, set by the protocol. The schedules that we calculate guarantee the absence of packet collisions, when the scheduling is repeated infinitely and periodically.

Several working groups are proposing technical solutions (presented in chapter~\ref{chap:TSN}) to help control latency in networks. With these solutions, network equipments are able to allocate transmission resources for certain flows at a given time. Research at Nokia Bell Labs aims to go further, by being able to reserve part of the resources at a given time for a given packet. However, it is necessary to calculate the dates on which packets must arrive at the network nodes. This thesis focuses on scheduling packets so that they do not collide in any of the nodes, in order to remove contention buffers. Completely get rid of contention buffers is not always possible, especially when networks are composed of many nodes. In this case, the goal of our work is to minimize the time spent by packets in the contention buffers. It is important to note that in this case, the waiting time in contention buffers is no longer undergone as in statistical multiplexing, but is controlled and predictable.

We model a network by a directed weighted acyclic multigraph whose vertices represent the contention points between messages in the network. The weights of the arcs represent the physical transmission time between two contention points. Two messages conflict if they must pass through the same contention point at the same time. We consider that routing is given, and we try to organize the messages so that there is no conflict in the network. We first study the problem on simple and common networks with two serial contention points. 
We define the decision problem of choosing the date each message passes through each of these two contention points so that no message conflicts with another in the network. This problem looks like classical scheduling problems, but the periodic sending of our flows makes it an original and difficult problem. We prove in chapter~\ref{chap:model} that the problem is $\NP$-complete, even on acyclic oriented graphs of low degree or low depth, by reducing problems of arc or vertex coloring. We therefore propose heuristics (glutton algorithms, metaheuristics) that allow us to find good solutions in polynomial time for any type of instance and FPT algorithms (of exponential complexity in the number of routes, but not in the other parameters of the problem) that find an optimal solution to the problem and that are fast enough on the simple instances we study.


We study in chapter~\ref{chap:PAZL} the problem of organizing non-synchronized flows in a network without any buffer.
Even if for the moment C-RAN-related protocols do not allow to desynchronize antennas (which could be the case for future generations of mobile networks), this approach is applicable in other contexts, such as a factory where robots that do not require synchronization need to communicate quickly with a control center.
We aim to compute a sending dates for messages at the beginning of their route so that they do not conflict with other messages. This sending time is not considered as contention time.
The solutions to this problem are all optimal in terms of latency: the latency of messages is equal to the physical time of transmission, because no latency is added to messages because of contention. We describe increasingly advanced greedy algorithms aimed at optimizing the impact of adding a message to the partial solution we computed. These algorithms allow us to ensure that a solution to the problem exists when the network load is less than $40\%$ (and even up to $61\%$ for messages of size $1$). We also propose an FPT algorithm (when the problem is parameterized by the number of routes) which allows us to compute the optimal solution in a reasonable time when the number of routes is less than $20$. Our results show that the problem cannot be solved without contention buffers when the network load is higher than $80\%$.
This is why we deal in chapter~\ref{chap:PALL} with the problem of organizing flows with one contention buffer on the route, in order to provide a greater degree of freedom. We study more particularly the problem of minimization, that is, finding a solution that minimizes the maximum latency of the routes. We propose a two-stage approach. First, we choose the sending time for messages such that there is no conflict on the first contention point, and then we solve the problem of choosing the waiting time for each message on the second contention point. To do so, we describe a polynomial algorithm based on the classical scheduling problem of the literature, adapted for periodicity. We also propose an FPT algorithm based on the same principle, but which guarantees to find the optimal solution. We show that we are able to find solutions for which latency is minimal for $99.9\%$ of instances in highly loaded networks, and that our methods give excellent results compared to statistical multiplexing.

In chapter~\ref{chap:SPALL}, we study the problem of organizing synchronized messages on any type of DAG. In this case, all the messages are sent at the same time by the sources and we allow the messages to wait in buffers at each contention point of the network. We study the problem of minimizing the maximum latency in the network. We start by describing greedy algorithms that find a realisable solution for any load, which are used as a starting point for the local search algorithms used later. We introduce a compact form of the problem that allows us to define a notion of neighborhood between the solutions in order to explore all of them. We study the performance of local search algorithms (hill-climbing, tabu search, simulated annealing) and we propose a Branch and Bound algorithm that lists all the solutions in a compact form, making enough cuts to find the optimal solution quickly. We show experimentally that the Branch and Bound algorithm is able to find an optimal solution in a reasonable time for $12$ routes, while simulated annealing allows to find much better solutions than statistical multiplexing for any instance size.

We then study, in chapter~\ref{chap:BE}, the impact of our scheduling algorithms when periodic and high-priority C-RAN flows share the network with Best-Effort, non-priority flows whose arrivals follow a stochastic process. We propose a method for adapting our algorithms that smooth the load of the C-RAN flows all over the period, without increasing latency. Our experiments show that, even if organizing the flows in a deterministic way as we do requires using a bit more bandwidth to reserve resources, the average latency of Best-Effort flows is better than with statistical multiplexing. We also show the same kind of results in an optical ring where the scheduling of C-RAN flows is made trivial by the technical opto-electronic conversion constraints. 

All our approaches are based on strong technical assumptions: the flows must be perfectly synchronized, the network must have a global controller and must be programmable. Chapter~\ref{chap:TSN} reviews the recently developed standards that are close to our hypotheses. We also show the limits of these standards, and we introduce equipment in the development phase that would enable us to reduce latency in networks to the physical transmission time.
