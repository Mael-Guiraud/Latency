%!TEX root = Manuscript.tex

\chapter*{Présentation de la thèse}
\label{chap:intro}
\minitoc

Les travaux présentés dans cette thèse s'inscrivent dans le contexte du développement de la 5G, et sont plus particulièrement axés sur la réduction de la latence dans les réseaux coeur opérateur.
L'un des objectifs pour la 5G est de garantir une latence bout en bout la plus faible possible, afin de pouvoir développer des applications pour lesquelles le temps de réponse est critique (vehicules autonomes, industrie 4.0, etc...)
Le cas d'application que nous étudions est le C-RAN (pour Cloud Radio Acess Network). Le but du C-RAN est de centraliser les unités de calculs situés aux pieds des antennes dans un ou plusieurs centres de calculs, afin de faciliter la maintenance et de réduire les couts d'exploitations. Les antennes envoient periodiquement des messages aux centres de calculs, qui calculent une réponse et l'envoient aux antennes avec la même fréquence. Le temps écoulé entre l'envoi d'un message par une antenne et la réception de sa réponse doit être inferieur à 3ms, une contrainte imposée par le protocol de communication radio (HARQ). Ces messages envoyés sont très lourds, et utilisent donc beaucoup de bande passante dans les réseaux.

La gestion actuelle des réseaux, le multiplexage statistique consiste à dimentioner les liens de façon à ce que le flux moyen de message utilisant ces liens puisse passer en même temps. Quand trop de paquets doivent utiliser un lien en même temps, on parle de contention et une partie d'entre eux est mise en file d'attente, que nous appelons buffer de contention. Faire attendre les messages dans ces buffers de contention augmente la latence des packets. Plus un réseau est chargé, plus la chance que les latences augmentent à cause de la contention. Les réseaux C-RAN envoyant une grande quantité de messages necéssitant une fable latence sont donc incompatibles avec la notion de multiplexage statistique.

Plusieurs groupes de travails proposent aujourd'hui des solutions techniques (présentées dans le chapitre~\ref{chap:context}) pour aider à controler la latence dans les réseaux. Avec ces solutions, les equipements du réseaux sont capable de reserver une partie des ressources à un temps donné pour un paquet donné. Il faut toutefois calculer les temps auquels les paquets doivent arriver dans les noeuds du réseau. Cette thèse se concentre sur le fait d'organiser les paquets, de façon à ceux qu'ils n'entrent pas en collisions dans aucun des noeuds, dans le but de supprimer les buffers de contention. Se passer completement des buffers de contention n'est pas toujours possible, notamment lorsque les réseaux sont composés de beaucoup de noeuds. Dans ce cas, l'objectif de nos travaux est de minimiser le temps passé par les paquets dans les files d'attentes. Il est important de souligner que dans ce cas la, les buffers de contentions ne sont plus subis comme pour le multiplexage statistique, mais controlé.

\chapter*{Introduction}
\label{chap:intro}
\minitoc


