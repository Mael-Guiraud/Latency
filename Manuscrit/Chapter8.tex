%!TEX root = Manuscript.tex

\chapter*{Conclusion}
\label{chap:concl}
\addcontentsline{toc}{chapter}{Conclusion}

In this thesis, we presented the problem of minimizing latency of periodic flows in a packet switched network. Current networks in use for telecommunication are based on statistical multiplexing: the links are dimensioned considering the average bit rate of the flows so that the flows can share a link most of the time or be buffered until enough capacity is available. Statistical multiplexing is a low-cost solution to deploy a network, but it does not guarantee the latency of the packets using it. If a burst of data is sent by one flow, shared resources become critical and some packets are buffered while waiting for their availability. These buffers are called contention buffers, and are a major source of latency.

We study the Cloud-RAN application case. In C-RAN, radio antennas periodically send packets to datacenters, that compute an answer and send it back to the antennas. The packets must have an end-to-end latency lower than a maximal value, required by the protocols. Statistical multiplexing is not able to guarantee an end-to-end latency for packets, and because of contention buffers, the more loaded is a network, the largest is the latency. In our C-RAN usecase, the flows are periodic and a large amount of data is sent at each period by the antennas and the datacenters. Thus, managing the packets with statistical multiplexing is not conceivable.

Several working groups (DetNet, TSN, see Chapter~\ref{chap:TSN}), have developed standards and mechanisms ensuring an upper bound on the latency in packet switched networks. The network devices can reserve a port during a given time to forward the traffic of a given flow without contention buffer. The arrival date of the packets must then be known and precise. To do so, a scheduling of every output port of the devices is computed ahead. Current approaches to compute this scheduling are based on stochastic laws, since most of the internet traffic follows a stochastic behavior. In such a situation, it is impossible to totally get rid of contention buffers. Nevertheless, since our flows are deterministic (the amount of data and the periodicity of the packets remains the same all over time), we show that deterministic scheduling guarantees a minimal latency of deterministic flows and also helps to reduce the latency of stochastic best-effort flows. 
Furthermore, TSN and DetNet mechanisms induce several sources of additional latency, like guard time around packets to prevent the time shift between clocks or buffering time of the header of the packets in every switch to read the destination. In this thesis, we go further by proposing solutions to reduce the end-to-end latency of packets to its physical transmission time by proposing a new generation of switch that get rid of technical constraints imposed by a statistical vision of the network.


 This thesis focuses on the problem of computing a deterministic scheduling for periodic flows, a problem that we prove to be $\NP$-hard for arbitrary networks.
 We study in Chapters~\ref{chap:PAZL},\ref{chap:PALL} this problem when the flows are unsynchronized, that is, we can choose the emission date of the packets in the sources of the flows. This case does not perfectly match with Cloud-RAN, but it corresponds to various use cases, like industry 4.0, autonomous vehicle\dots
In Chapter~\ref{chap:PAZL}, we give several greedy algorithms and one FPT algorithm that allows to reduce the latency to the physical transmission time (i.e. without any contention) on a common network topology with a single shared link, when the load induced by the deterministic traffic is low enough. We experimentally show using the exact FPT algorithm, that on very loaded networks (when the load is greater than $80\%$), it is not possible to get rid of contention buffers. 
We then propose solutions that buffer packets in nodes of the network, but we try to minimize this additional latency. Remark that in such an approach, the buffers are not anymore a consequence of contention that we cannot control or predict, but a tool to organize the packets.
Chapter~\ref{chap:PALL} study the problem of organizing flows in a network with a single shared link (as in Chapter~\ref{chap:PAZL}), and allowing one buffer (positionned in the datacenters) on the route for every packet. We propose several greedy algorithms and one FPT algorithm based on a classical scheduling algorithm that we adapted for periodicity. The performances of our algorithm are excellent, we show it is possible to reduce latency to the physical transmission time of the longest route in $99,9\%$ of the cases, while statistical multiplexing, even prioritizing critical flows adds a latency to the flows, due to contention buffers, equal to $1/4$ of the period.

We study in Chapter~\ref{chap:SPALL} the C-RAN use case, in which all antennas send their messages at the same date, on arbitrary networks. We propose a compact form of the solutions to our problem, that allows to define a neighborood of a solution. Then, several local search heuristics are designed using this notion of neighborood. A branch and bound algorithm based 
on the compact form of the solutions is also proposed and run efficiently for small C-RAN network with ten to twenty routes. Then, we experimentally show that our approach dramatically over performs statistical multiplexing in terms of latency.

We then show in Chapter~\ref{chap:BE} how to adapt our algorithm to not impact best-effort flows latency while scheduling the C-RAN traffic. We explain how to adapt our algorithms by computing with artificialy incresed values of message size, that do not impact the latency of C-RAN datagrams, and smooth the load of C-RAN traffic all over the period, 
 We show that, even if our approaches induces an additional use of bandwidth due to resource reservation, we are able to improve the average latency of best-effort traffic while minimizing the C-RAN traffic latency. We also show similar results in an industrial optical ring, in which scheduling the C-RAN packets is trivial because of the architecture.

We proposed solutions to minimize the end-to-end latency in various use cases: Cloud-RAN, Industry 4.0, motion control, autonomous vehicle, etc\ldots
Reducing the transmission latency allows to:
\begin{itemize}
	\item Respect latency constraints required by protocols
	\item Increase the Network Quality of Service 
	\item Allow more time to the other components of the chain (computation in datacenters for C-RAN example) 
	\item Lengthen the physical links, which means, for C-RAN a wider area of development and thus to lower exploitation and development costs (CAPEX, OPEX).
\end{itemize} 

\section*{Further researchs}

Several questions remains open around our work. Even if we conjecture that \pazl and \pall are $\NP$-hard on star routed networks, we are not yet able to prove it. 
The algorithms we developed for \pazl and \pall are designed for star routed networks. Even if several greedy algorithms could be adapted for arbitrary networks, we did not study them in detail. Furthermore, the FPT algorithms are based on the fact there is no or only one buffers in the networks, and are not easily adaptable.

This thesis arises in the context of SDN which aims to develop dynamical and programmable networks. In C-RAN for example, the radio network aims to be able to turn off antennas when the number of connected devices is low. Our algorithms for \pazl and \pall compute the scheduling for all flows, and must reschedule the entire solution if a flow is removed or added to the network to ensure a minimal latency. In the case of the algorithms presented for \minstra, the compact form of the solutions we presented allows us to efficiently add a flow to the best solution, but not to quickly re-compute the best solution when a flow is removed.

The measure we want to minimize is the maximal end-to-end latency of all flows. This often means that the flow using the longest route is never delayed in contention buffer, but the other flows are more or less impacted by the solution we give. One can imagine use-cases in which all flows do not share the same latency constraints, and define another problem in which we try to minimize the latency of all flows independently. We can also expand our model to consider several kind of flows, each one with a different period, size of message and latency constraint. 



