

\begin{titlepage}


%\thispagestyle{empty}

\newgeometry{left=7.5cm,bottom=2cm, top=1cm, right=1cm}

\tikz[remember picture,overlay] \node[opacity=1,inner sep=0pt] at (-28mm,-135mm){\includegraphics{Bandeau_UPaS.pdf}};

% fonte sans empattement pour la page de titre
\fontfamily{fvs}\fontseries{m}\selectfont


%*****************************************************
%******** NUMÉRO D'ORDRE DE LA THÈSE À COMPLÉTER *****
%******** POUR LE SECOND DÉPOT                   *****
%*****************************************************

\color{white}

\begin{picture}(0,0)

\put(-150,-735){\rotatebox{90}{NNT: 2020UPASA000}}
\end{picture}
 
%*****************************************************
%**  LOGO  ÉTABLISSEMENT PARTENAIRE SI COTUTELLE
%**  CHANGER L'IMAGE PAR DÉFAUT **
%*****************************************************
\vspace{-10mm} % à ajuster en fonction de la hauteur du logo



%*****************************************************
%******************** TITRE **************************
%*****************************************************
\flushright
\vspace{10mm} % à régler éventuellement
\color{Prune}
\fontfamily{fvs}\fontseries{m}\fontsize{22}{26}\selectfont
 Deterministic architectures for low latency in 5G and beyond systems

%*****************************************************

%\fontfamily{fvs}\fontseries{m}\fontsize{8}{12}\selectfont
\normalsize
\vspace{1.5cm}

\color{black}
\textbf{Thèse de doctorat de l'Université Paris-Saclay}

\vspace{15mm}

École doctorale n$^{\circ}$ 000, dénomination et sigle\\
\small Spécialité de doctorat: voir annexe\\
\footnotesize Unité de recherche: voir annexe\\
\footnotesize Référent: : voir annexe
\vspace{15mm}

\textbf{Thèse présentée et soutenue à ....., le .... 202X, par}\\
\bigskip
\Large {\color{Prune} \textbf{Maël Guiraud}}


%************************************
\vspace{\fill} % ALIGNER LE TABLEAU EN BAS DE PAGE
%************************************

\flushleft \small \textbf{Composition du jury:}
\bigskip



\scriptsize
\begin{tabular}{|p{8cm}l}
\arrayrulecolor{Prune}
\textbf{Prénom Nom} &   Président/e\\ 
Titre, Affiliation & \\
\textbf{Prénom Nom} &  Rapportrice \\ 
Titre, Affiliation   &   \\ 
\textbf{Prénom Nom} &  Rapporteur \\ 
Titre, Affiliation  &   \\ 
\textbf{Prénom Nom} &  Examinatrice \\ 
Titre, Affiliation   &   \\ 
\textbf{Prénom Nom} &  Examinateur \\ 
Titre, Affiliation   &   \\ 
\textbf{Prénom Nom} &  Examinateur \\ 
Titre, Affiliationt   &   \\ 

\end{tabular} 

\medskip
\begin{tabular}{|p{8cm}l}\arrayrulecolor{white}
\textbf{Dominique Barth} &   Directeur\\ 
Professeur, UVSQ & \\
\textbf{Yann Strozecki} &   Coencadrant\\ 
Maître de conférence, UVSQ  &   \\ 
\textbf{Olivier Marcé} &  Coencadrant \\ 
Ingénieur de recherche  & Nokia Bell Labs France  \\ 
\textbf{Brice Leclerc} &  Coencadrant \\ 
Ingénieur de recherche  & Nokia Bell Labs France  \\ 

\end{tabular} 


\end{titlepage}
%%%%%%%%%%%%%%%%%%%%%%%%%%%%%%%%%%%%%%%%%%%%%%%%%%%%%%%%%%%%%%%
% 4eme de couverture
\ifthispageodd{\newpage\thispagestyle{empty}\null\newpage}{}
\thispagestyle{empty}
\newgeometry{top=1.5cm, bottom=1.25cm, left=2cm, right=2cm}
\fontfamily{rm}\selectfont

\lhead{}
\rhead{}
\rfoot{}
\cfoot{}
\lfoot{}

\noindent 
%*****************************************************
%***** LOGO DE L'ED À CHANGER ÉVENTUELLEMENT *********
%*****************************************************
\includegraphics[height=2.45cm]{EOBE}
\vspace{1cm}
%*****************************************************

\begin{mdframed}[linecolor=Prune,linewidth=1]
\vspace{-.25cm}
\paragraph*{Titre:} Ordonnancement periodiques de messages pour minimiser la latence dans les réseaux dans un contexte 5G et au delà

\begin{small}
\vspace{-.25cm}
\paragraph*{Mots clés:} Theorie des graphes, Cloud Radio Access Network

\vspace{-.5cm}
\begin{multicols}{2}
\paragraph*{Résumé:} todo 
\end{multicols}
\end{small}
\end{mdframed}

\begin{mdframed}[linecolor=Prune,linewidth=1]
\vspace{-.25cm}
\paragraph*{Title:} Deterministic and periodic datagrams scheduling for low latency in 5G and beyond

\begin{small}
\vspace{-.25cm}
\paragraph*{Keywords:}  Graph theory, Cloud Radio Access Network

\vspace{-.5cm}
\begin{multicols}{2}
\paragraph*{Abstract:} Cloud-RAN (C-RAN) is an architecture for cellular networks, where processing units, previously attached to antennas, are centralized in data-centers. The main challenge, to fulfill protocol time constraints, is to minimize the latency of the periodic messages sent from the antennas to their processing units and back. We show that statistical multiplexing suffers from high logical latency, due to buffering at nodes to avoid to collisions. Hence, we propose to use a \emph{deterministic} scheme for sending periodic messages \emph{without collision} in the network thus saving the latency incurred by buffering.

We give several algorithms to compute such schemes. We first study a common topology where one link is shared by all antennas. We show there is a solution without any buffering when the routes are short or the load is small. When the parameters are unconstrained, and buffering is allowed in processing units, we propose \PMLS algorithm adapted from a classical scheduling method. Experimental results show that even under full load,  \PMLS finds a deterministic sending scheme with no logical latency most of the time, while the classical solution of statistical multiplexing adds a very large latency. Using a modification of this algorithm, we obtain very low latency periodic sending schemes which do not disrupt additional random traffic on the network.

We then study a similar problem on general topologies when the flows are synchronized, that is, all the antennas must send their message at the same date. We show that the problem is NP-hard and propose an FPT algorithm, exponential on the lowest parameter of the problem: the number of flows. We also propose some local search heuristics that give good solutions.

Finally, we show how the model is close from real emerging technologies (TSN) and we study the impact on our determnistic schemes on best effort traffic, on every topologies, including an optical ring studied in the european project N-GREEN.
\end{multicols}
\end{small}
\end{mdframed}

%************************************
\vspace{3cm} % ALIGNER EN BAS DE PAGE
%************************************
\fontfamily{fvs}\fontseries{m}\selectfont
\begin{tabular}{p{14cm}r}
\multirow{3}{16cm}[+0mm]{{\color{Prune} Université Paris-Saclay\\
Espace Technologique / Immeuble Discovery\\
Route de l’Orme aux Merisiers RD 128 / 91190 Saint-Aubin, France}} & \\
\end{tabular}
