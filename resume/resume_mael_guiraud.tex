%Compiler avec pdflatex, le résultat ne doit pas dépasser une page
%Merci de respecter scrupuleusement ce modèle.
\documentclass[a4paper,titlepage,12pt,normalheadings,makeidx]{article}

\usepackage{graphicx} 
\usepackage[utf8]{inputenc}
\usepackage[francais]{babel}


\begin{document}
\newpage

% Titre
\subsection*{Gestion de la contention pour les réseaux 5G}

% Sommaire
\addcontentsline{toc}{section}{Gestion de la contention pour les réseaux 5G}

% Liste des auteurs (l'orateur est souligné) 
Dominique Barth, DAVID, Versailles,
\texttt{dominique.barth@uvsq.fr}\\
\indent
Yann Strozecki, DAVID, Versailles,
\texttt{yann.strozecki@uvsq.fr}\\
\indent
Olivier Marcé, NOKIA BELL LABS, NOZAY,
\texttt{Olivier.Marce@nokia-bell-labs.com}\\
\indent
\underline{Maël Guiraud}, DAVID, Versailles,
\texttt{mael.guiraud@ens.uvsq.fr}\\
\\

% Corps du résumé
L'architecture des équipements pour les réseaux 5G vise à centraliser les unités de calcul des antennes radios vers un ou plusieurs centres de calculs communs. Le réseau créé entre les antennes et ces centres de calculs est donné, et on veut y envoyer de façon périodique des paquets depuis les antennes de façon à ce que les réponses arrivent dans un très court délai (3ms). On doit donc contrôler la latence. Sachant que le routage du réseau nous est imposé, nous devons choisir les dates d'envoi des messages depuis les antennes de façon à éviter les contentions dans les \textit{switchs}.  \\

Nous modélisons un réseau par un graphe, dans lequel il y a deux ensembles distincts de sommets, représentant les antennes (A) et les unités de calculs (UC). Les autres sommets du graphe représentent les nœuds du réseau. Le poids sur les arrêtes correspond au délai physique de transmission d'un message dans les liens.
On peut modéliser les conflits entre les différentes routes du réseau par le graphe des conflits. Ses sommets sont les routes et il y a une arrête entre deux sommets si les routes correspondantes se croisent.
Les arrêtes sont pondérées en fonction de la taille des deux routes avant leur point de conflit.
Résoudre notre problème revient à trouver un type de coloriage (que nous avons introduit) du graphe des conflits, ce qui à été montré NP-Complet.

Le phénomène que nous étudions est un aller retour dans un réseau. On considère un graphe, un routage dans ce graphe et une période. Le but est de trouver, en cette période, une date de départs des paquets de façon à ce qu'il n'y ait pas de collision, dans chacun des sommets du graphe et qu'aucun aller-retour n'ai un temps de trajet supérieur à la \textit{deadline}.\\

Nous avons, dans un premier temps, étudié le problème sur une topologie simple : une étoile. Sur cette topologie, nous avons imaginé plusieurs algorithmes, dont certains donnent des résultats optimaux quand la taille des paquets est inférieure à la taille des routes. Les résultats obtenus montrent que l'approche déterministe garantie des latences bien inférieures à celles obtenues avec une approche statistique, utilisée généralement pour gérer les réseaux.

\vfill
\end{document}
