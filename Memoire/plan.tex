\documentclass[a4paper,10pt,openany]{book}

\usepackage[utf8]{inputenc}

\begin{document}
\begin{chapter}{Context, company and subject}
\begin{section}{Company}
 Description de Nokia Bell Labs, Activité dans les équipements télécoms. (Voir avec Olivier)
\end{section}
\begin{section}{Context}
 Utiliser le sujet de these : C-RAN pour 5g. Centralisation des BBU à travers les réseaux fronthaul.
 Fortes contraintes en débit et en latence. Expliquer le sujet : trouver des algorithme qui permettent de faire 
 transiter les messages sans contentions dans les switchs.
\end{section}

\end{chapter}

\begin{chapter}{Problem, model}
 \begin{section}{Model}
  Description de la topologie du réseau, du modèle (un réseau = un graphe, noeuds = BBU RRH ou SWITCHS), 
  définitions des objets utilisés dans la modélisation. Parler des temps slottés.
  Définir les 3 topologies et expliquer qu'on ne s'interesse qu'a la première

 \end{section}
 \begin{section}{Problem}
   Bien définir le problème :  contrainte de periodicité de 1ms, et on cherche a réduire le temps aller-retour pour passer
  en dessous des 3ms. 
 \end{section}
 \begin{section}{Related Work}
Résumé des articles Simons et Path coloring/Call Scheduling.
Parler du passage du Garey and Johnson sur la complexité.
 \end{section}


\end{chapter}

\begin{chapter}{Algorithms and studies }


\begin{section}{First approach : without waiting times}
 Présentation des différents algos (celui avec des certitudes théoriques, le bruteforce et l'algo sur lequel on n'a pas d'analyse) pour trouver 
 une solution sans waiting times. Ne pas hesiter a mettre les exemples de l'article qui prouvent que le waiting time est nécéssaire avec notre contrainte de temps.
 \end{section}

\begin{section}{Main Algorithm}
Description du fonctionnement de l'algo Simons sur notre modèle. 
 Résultats théoriques sur les différents cas de la topologie 1. Ici expliquer que les cas 1 et 2 sont triviaux, et qu'on ne traite désormais que le 
 cas global.
\end{section}

\end{chapter}

\begin{chapter}{Simulations}


\begin{section}{Pour chaque courbe}
\begin{itemize}
 \item Données en entrées
 \item Métrique, donnée mesurée, algos comparés
 \item Resultats, commentaires infos importante que l'on conclus.
\end{itemize}
\end{section}

\begin{section}{No waiting time involve bigger periods}
Ici montrer le graph de la taille des fenêtres pour les algos sans waiting time.
 En conclure que le bruteforce est le mieux
\end{section}

\begin{section}{Limits of ``Nom que je donne au bruteforce''}

 Résultats de simulations qui montrent à partir de quelle charge du réseau le bruteforce nous indique qu'il n'y 
 a plus de solutions sans waiting time.
Conclusion : utilité d'algos dans lesquels on autorise les waiting times
\end{section}

\begin{section}{Longest Shortest}
Graphique des $T_{max}$ pour l'algo longest shortest et random.
Montrer que les résultats sont très bon pour longest shortest et très mauvais pour random dès que le charge dépasse un certain seuil.
Montrer aussi les distributions cumulées qui montrent une nouvelle fois que le longest shortest est utile.
\end{section}

\begin{section}{Synthesis}
Reprendre tous les résultats
\end{section}

\end{chapter}



\begin{chapter}{Conclusion}
\begin{itemize}
 \item Comprendre le problème (parler de l'article qui à été faite avant par DAVID sur le sujet)
 \item Lecture des articles, définition du modèle
 \item Simulations
 \item résultats, utilisables sur les autres topologies ?...
\end{itemize}



\end{chapter}
\end{document}
