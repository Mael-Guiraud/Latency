\documentclass[a4paper,10pt]{article}
\usepackage[utf8]{inputenc}

\usepackage[backend=bibtex,style=verbose]{biblatex}
\addbibresource{Sources.bib}


\begin{document}
   
  \cite{chitimalla20175g}
  
 
Cet article présente une problématique très similaire à la notre, sur une architecture "star shaped". Le problème regardé est de minimiser le jitter sur des flux C-RAN. Contrairement à nous, les auteurs ne traitent que l'aller et non le retour. En revanche, ils s'autorisent plusieurs debits différents de C-RAN, ce que nos algorithmes à l'heure actuelle ne peuvent pas supporter.
L'algorithm proposé est une routine greedy qui fixe les messages dans la période par ordre priorité sur les routes. Cette routine est appliquée à tous les ordres possible de routes et donc donne un algo exponentiel.
Les résultats traitent d'une part du choix du split et de son impact sur la portée pouvant être couverte : trade off entre un split "lourd" qui a des fortes contraintes en latence mais qui permet de ne pas perdre de temps dans la partie technique (encapsulation), et un split "leger" qui a les caractéristiques contraire (longue encapsulation mais faible contraintes en latence donc distance plus grande).
Dans la deuxième partie, les auteurs évaluent leur algo et expliquent qu'ils peuvent faire du Zero jitter sur des réseaux peu chargé.

A retenir de cet article : les citations au debut, des données et des explications interessantes sur le CPRI.
    


\end{document}