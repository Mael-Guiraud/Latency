%\documentclass[a4paper,10pt]{article}
\documentclass[10pt, conference, letterpaper]{IEEEtran}
\usepackage[utf8]{inputenc}
\usepackage{xspace}
\usepackage{url}
\usepackage{graphicx,graphics} 
\usepackage{color}
\usepackage{amsmath}
\usepackage{amsfonts}
\usepackage{amssymb}
\usepackage{amsthm}
\usepackage{algorithm}
\usepackage{algorithmic}
\usepackage{longtable}
\usepackage{complexity}
\usepackage{tkz-graph}
\usepackage{float}
\usepackage{tabularx}
\usepackage{setspace}
\usepackage{icomma}
\renewcommand{\algorithmicrequire}{\textbf{Input:}}
\renewcommand{\algorithmicensure}{\textbf{Output:}}
\usepackage{authblk}
\usepackage[colorlinks=true,breaklinks=true,linkcolor=blue]{hyperref}


\newcommand\rmatching{${\cal R}$-matching\xspace}
\newcommand\mdelay{$\cal M$-delay\xspace}
\newcommand\matchedgraph{{\bf matched graph}}
\newtheorem{proposition}{Proposition}
\newtheorem{theorem}{Theorem}

\setlength{\parskip}{1ex} % Espace entre les paragraphes

\newtheorem{fact}{Fact}
\newtheorem{lemma}[theorem]{Lemma}
\newtheorem{definition}{Definition}
\newtheorem{corollary}{Corollary}

% \renewcommand{\thefootnote}{\*}

\newcommand{\todo}[1]{{\color{red} TODO: {#1}}}
\newcommand\pazl{\textsc{pazl}\xspace}
\newcommand\pall{\textsc{pall}\xspace}
\newcommand\bra{\textsc{bra}\xspace}
\newcommand\pra{\textsc{pra}\xspace}
\newcommand\minpra{\textsc{min-pra}\xspace}
%opening
\title{Greedy algorithms for scheduling periodic message }
 

% \author[1]{Dominique Barth}
\author[1,2]{Ma\"el Guiraud}
% \author[1]{Christian Cad\'er\'e}
% \author[2]{Brice Leclerc}
% \author[2]{Olivier Marc\'e}
\author[1]{Yann Strozecki}
\affil[1]{David Laboratory, UVSQ}
\affil[2]{Nokia Bell Labs France}

\begin{document}

\maketitle

\begin{abstract}

A recent trend in mobile networks is to centralize in distant data-centers  processing units which were attached to antennas until now. The main challenge is to guarantee that the latency of the periodic messages sent from the antennas to their processing units and back, fulfills protocol time constraints. The problem is then to propose a sending scheme from the antennas to their processing units and back without contention and buffer.

We study a star shaped topology, where all contentions are on a single arc shared by all antennas. We present several greedy heuristic to solve \pazl. We study their experimental efficiency and we use them to prove that when the load of the network is less than $44\%$, there is always a solution to \pazl. We also prove that for random lengths of the arcs, most of the instances have a solution when the load is less than $45\%$.  
\end{abstract}


\section{Introduction}


\section{Model}
Describe the general model, and objective to optimize: no buffering.
It is a scheduling/packing problem, could call each element a task or just an element. 
Notion of depth.

 + the simplified version for the star with a picture + a set of number $d_1,\dots,d_n$ and two values $P, \tau$. Notion of load.
\cite{Guir1806:Deterministic}

\section{Basic Algorithms}

Notion of partial solution, how to extend it (and notation). All
algorithms are greedy and most can work online. Once a route is placed,
it is not changed, hence in a partial solution we are only interested in the 
set of positions used in the backward and forward period. If $n$ tasks has been
placed, we denote them by $x_1, \dots, x_n$ and $y_1,\dots,y_n$.

\subsection{Depth one}

Algorithm of increasing quality

First fit, analyzed naively $n/4$, then better through compacity $n/3$.

Meta interval, load of $n/3$.

First fit in order of the $d_i \mod \tau$ $n/2$, as good as naive first fit for 
$\tau = 1$ (all alg degenerate to this one for $\tau =1$).

\subsection{General graph}

Use the coherent routing property.
Algo first fit :$n/4$ for any $\tau$ and $P$, and any DAG of depth $k$. 

Copy the results of the previous article and propose heuristics to chose among several positions/candidates (compacity heuristic).

\subsection{Experimental results}

Results better in practice: give the data
Two heuristics to test:
\begin{itemize}
	\item  heuristic to build super compact assignment (among the compact assignments
possible, chose the one which maximize the gain on the second bloc)
	\item heuristic to maximize the free position of the remaining elements
\end{itemize}

Quality of the results explained by the average analysis done later
 

\section{Why $\tau$ can be assumed to be one}

Rank the $d_i$ by value, and compute $d_i + d_n \mod \tau$.
We allow buffering, but the worst time should not increase !
Bufferize  each route during $\tau - (d_i + d_n \mod \tau)$.
All routes have the same remainder mod $\tau$, can assume they are of 
size one.
A bit mor complex on the general graphs, proofs on the depth of the graph. 
Should take into account the length of the graph.


\section{Above $1/2$}

We show that we can go above $1/2$ using a greedy algorithm.

TODO: properly define the potential of a position, of a route. Relates both through a lemma 
showing that the sum of both is equal. 
Also a lemma which relates potential to the number of conflicts. 

We assume that all distances are different (if at least a constant fraction are, it is ok).
The greedy algorithm is as follows, for $(1/2 + \epsilon)P$ routes:

Let $S$ be the set of routes of potential $2\epsilon n$, intialized to $\emptyset$.
When a route is added to $S$, it is removed from the routes used to compute the potential. 
When $|S| \geq \epsilon n$, the algorithm stops.
For $k < P/4$ routes already placed, the potential of a position $i$ on the backward windows 
is the number of yet unused routes of delay $d$ such that $i-d \mod P$ is used.
This potential corresponds to the number of routes for which placing a route at this position
remove only one possible offset instead of two.
On the backward windows select the unused position of largest potential and find a route which can be 
placed at this position, not in the best $\epsilon n - |S|$ routes.  

Proof it works: 

Since all distances are differents, 
there are at least $(1/2 + \epsilon)n - |S|- k$ positions in the forward windows which allows
to place a route attaining a given position in the backward windows. 
Since there are $k$ used offsets in the forward windows, there are at least $(1/2 + \epsilon)n - |S|- 2k$ free positions among them. Since $k < n/4$, there are at least $\epsilon n - |S|$ possible routes satisfying the constraint, hence one is not among the $\epsilon n - |S|$ routes having best potential
(with the chosen variant, not useful to spare the best ones, simplifiyng the proof and the algorithm).


Assume that there is a free position, wich increases the global potential by at least $2/3$
of the average which is $k(1/2 - k)/n$. Then the global potential is a least $(2/3)\sum_{i=1}^{k}i(1/2 - i)/n$. On the other hand, if no position of at least $2/3$ the average is avalaible, then the global
potential is at least $1/3 k(1/2 - k)$. (to clean that, prove by induction that the potential is always larger than some value at step $k$).
If we can prove that when $k = P/4$, the potential is at least $2\epsilon n (1/2 + \epsilon)n$, then 
it implies that the algorithm has stopped before that point and it implies there is a solution.
We solve the equation lower bound on the potential larger than $2\epsilon n (1/2 + \epsilon)n$
to obtain a correct value on $\epsilon$.



The computation is not optimised at all. We win only on the backward windows but 
we should win as much on the forward windows. Several simplification in the bound (but of little impact).
Lose too much when dealing with only positions of potential lower than the average (can do better than $1/3 $ vs $2/3$).

\section{Algorithms for random instances}

\paragraph{$\tau = 1$}

We analyze the following process, called \textbf{Uniform Greedy} or UG.
For each element in order, chose one admissible position
uniformly at random. We analyze the probability that Uniform Greedy
solve the problem, averaged over all possible instances. 
It turns out that this probability, for a fixed load strictly less than
one goes to zero when $m$ grows. 

Définir l'ensemble des solutions de taille $n$ parmi $m$.
\begin{theorem}
Given an instance of size $n$ uniformly at random UG
produces a solution uniformly at random or fail.
\end{theorem}
\begin{proof}
Regarder mes notes partielles pour compléter ça.
\end{proof}

Let us denote by $P(m,n)$ the probability that UG fails at the $n_th$
steps assuming it has not failed before.

\begin{theorem}
We have $$P(m,n) = \frac{\binom{n}{2n-m}}{\binom{m}{n}}.$$
In particular, $P(m,n) \leq f(\lambda)^m$, where $f(\lambda) < 1$.
\end{theorem}
\begin{proof}
Probability independent of the shift of the $n$ element, can say it is $0$.
It is the probability that two sets of size $n$ in $[m]$ are of union $[m]$.
It is the same as the probability that it contains a given set of size $m-n$.
Could find an asymptotic online.
\end{proof}

Can we make the same argument for a deterministic algorithm?
The not average version of the argument is the previous proof.

\section{Greedy and delay}

Tradeoff between waiting time and load. Can we prove $0.5 + \epsilon$ load for $f(\epsilon,n,\tau)$ waiting time ?  No idea yet.


\section{Non greedy algorithm}

Greedy + swap one element if necessary. 
Can we guarantee a solution for a larger load ? Conjecture, yes for $\lambda = 2/3$. 
Seems hard for group theoritic reason (how to avoid subgroups of Z/pZ which are a problem)

\section{Lower bounds}

Example/family of examples for which some greedy alg fail.
Example/family of examples with a given load such that there are no feasible solution.

\section{$\NP$-hardness}

Are we able to prove $\NP$-hardness ?

% \bibliographystyle{ieeetr}
% \bibliography{Sources}

\end{document}
