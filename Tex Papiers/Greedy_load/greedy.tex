%\documentclass[a4paper,10pt]{article}
\documentclass[10pt, conference, letterpaper]{IEEEtran}
\usepackage[utf8]{inputenc}
\usepackage{xspace}
\usepackage{url}
\usepackage{graphicx,graphics} 
\usepackage{color}
\usepackage{amsmath}
\usepackage{amsfonts}
\usepackage{amssymb}
\usepackage{amsthm}
\usepackage{algorithm}
\usepackage{algorithmic}
\usepackage{longtable}
\usepackage{complexity}
\usepackage{tkz-graph}
\usepackage{float}
\usepackage{tabularx}
\usepackage{setspace}
\usepackage{icomma}
\renewcommand{\algorithmicrequire}{\textbf{Input:}}
\renewcommand{\algorithmicensure}{\textbf{Output:}}
\usepackage{authblk}
\usepackage[colorlinks=true,breaklinks=true,linkcolor=blue]{hyperref}


\newcommand\rmatching{${\cal R}$-matching\xspace}
\newcommand\mdelay{$\cal M$-delay\xspace}
\newcommand\matchedgraph{{\bf matched graph}}
\newtheorem{proposition}{Proposition}
\newtheorem{theorem}{Theorem}

\setlength{\parskip}{1ex} % Espace entre les paragraphes

\newtheorem{fact}{Fact}
\newtheorem{lemma}[theorem]{Lemma}
\newtheorem{definition}{Definition}
\newtheorem{corollary}{Corollary}

% \renewcommand{\thefootnote}{\*}

\newcommand{\todo}[1]{{\color{red} TODO: {#1}}}
\newcommand\pazl{\textsc{pazl}\xspace}
\newcommand\pall{\textsc{pall}\xspace}
\newcommand\bra{\textsc{bra}\xspace}
\newcommand\pra{\textsc{pra}\xspace}
\newcommand\minpra{\textsc{min-pra}\xspace}
%opening
\title{Greedy algorithms for scheduling periodic message }
 

% \author[1]{Dominique Barth}
\author[1,2]{Ma\"el Guiraud}
% \author[1]{Christian Cad\'er\'e}
% \author[2]{Brice Leclerc}
% \author[2]{Olivier Marc\'e}
\author[1]{Yann Strozecki}
\affil[1]{David Laboratory, UVSQ}
\affil[2]{Nokia Bell Labs France}

\begin{document}

\maketitle

\begin{abstract}

A recent trend in mobile networks is to centralize in distant data-centers  processing units which were attached to antennas until now. The main challenge is to guarantee that the latency of the periodic messages sent from the antennas to their processing units and back, fulfills protocol time constraints. The problem is then to propose a sending scheme from the antennas to their processing units and back without contention and buffer.

We study a star shaped topology, where all contentions are on a single arc shared by all antennas. We present several greedy heuristic to solve \pazl. We study their experimental efficiency and we use them to prove that when the load of the network is less than $44\%$, there is always a solution to \pazl. We also prove that for random lengths of the arcs, most of the instances have a solution when the load is less than $45\%$.  
\end{abstract}


\section{Introduction}


\section{Model}
Describe the general model + the simplified version for the star with a picture + a set of number $d_1,\dots,d_n$ and two values $P, \tau$. 
\cite{Guir1806:Deterministic}

Explain the notion of shadow ? Give the compacity critieria.

\section{Basic Algorithms}

\subsection{General graph}

Algorithm with meta intervals with load $n/4$
Specialization to the result of the article for the star with $n/3$.

Copy the results of the previous article and propose heuristics to chose among several positions/candidates (compacity heuristic).

\subsection{First position}

Describes how it builds compact assignments 
+ heuristic to build super compact assignment (among the compact assignments
possible, chose the one which maximize the gain on the second bloc)

\section{Towards $n/2$}

Simple result with $\tau = 1$: $n/2$. 
Two questions:
\begin{itemize}
 \item can we do better with a greedy alg ?
+ add the results with the heuristic which reduce collisions with the last elements
by choosing properly the first $n/2$ elements which are placed to gain $\sqrt{n}$ elements. 
\item can we get the same bound for any $\tau$ ?
Answer: Yes by paying a waiting time of at most $\tau$ (all blocs aligned on meta interval,
degenerate to the case $\tau = 1$).
\end{itemize}

\subsection{Pairs of elements in order of shifts}

Precise description of the algorithm. 

\subsection{Tuples of elements in order of shifts}

Bit more sketchy description + value derived from the programm computing the value with many
different size of groups.

\section{Algorithms for random instances}

\paragraph{\tau > 1}
Compute a better bound working w.h.p. for some greedy algorithms (celui qui fait un ou plusieurs serpents). 

\paragraph{\tau = 1}

Greedy algorithm with constraints on the possible positions. 
At each step placing a new route take only one slot more and more often. 
Should give a $\2 - \sqrt{2} = 0.585\dots$ bound.

\section{Greedy and delay}

Tradeoff between waiting time and load. Can we prove $0.5 + \epsilon$ load for $f(\epsilon,n,\tau)$ waiting time ?  No idea yet.


\section{Non greedy algorithm}

Greedy + swap one element if necessary. 
Can we guarantee a solution for a larger load ? Conjecture, yes for $\lambda = 2/3$. 
Seems hard for group theoritic reason (how to avoid subgroups of Z/pZ which are a problem)

\section{Lower bounds}

Example/family of examples for which some greedy alg fail.
Example/family of examples with a given load such that there are no feasible solution.

\section{$\NP$-hardness}

Are we able to prove $\NP$-hardness

% \bibliographystyle{ieeetr}
% \bibliography{Sources}

\end{document}
