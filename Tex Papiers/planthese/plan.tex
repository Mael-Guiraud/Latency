\documentclass[a4paper,10pt]{article}
\usepackage[utf8]{inputenc}
\newcommand{\red}[1]{{\color{red} {#1}}}
%opening
\title{Plan de thèse}
\author{Maël}

\begin{document}

\maketitle

\begin{abstract}
Voici mon plan de thèse.
\end{abstract}
\section{Introduction}
Faut il un chapitre juste pour l'intro ?
Résumé  de tout ce qu'on va dire dans la section 2, puis plan de la these, et explication de ce qu'on a fait par rapport au problème.
\section{Définition du probleme}

\subsection{Notions utiles}
Définition des notions d'algo de graph.\\
Explications précises de l'algo de Simons, qui sert tout le temps \\
Mettre ca en annexe ?
A discuter ...
\subsection{Model}
\begin{itemize}
 \item Temps discrets
 \item Description du routed network (modèle avec les sommets qui sont des conflits), Ce que ca réprésente en Cloud-RAN.
 \item Periodicité
 \item conflict depth
 
\end{itemize}
\subsection{Problemes}
Definition de BRA,PALL, SPALL, Expliquer la différence entre les deux (exemple ou une solution optimale pour l'un ne l'est pas pour l'autre et inversement), les deux metriques sur chacuns des problèmes, et définition des problemes. NP-complétude pour chacun des problèmes, sur l'étoile ou en général.
Dire que PALL est moins contraint que SPALL.
\subsection{Contexte}
\subsubsection{Algo}
Expliquer ce qu'on a lu, pourquoi les models les plus proches sont différents, ou quelles sont les approches différentes que les gens ont eu.
\subsubsection{Reseau}
Expliquer les enjeux techniques
\begin{itemize}
 \item 5G
 \item URLLC
 \item C-RAN
 \item Detnet
 \item TSN
\end{itemize}

\section{Flux désynchronisés}
Ne peux pas s'impliquer dans le contexte C-RAN, plus pour une 6G ou eventuellement on pourait désynchroniser les flux. Sinon dire que ca pourrait être utile dans un contexte différent (usine ou autre)
Dans cette section on ne s'interesse qu'a des flux désynchronisés dans une topologie en etoile.
On peut peut être simplifier le model, ou au moins les notations juste pour cette partie, car on a pas besoin  des buffers.
Justifier l'envie de ne rien buffuriser par l'utilisation des réseaux optiques.
\subsection{Flux desynchronisés sans buffer}
Définition du problème PAZL, variante de PALL dans laquelle on ne veux vraiment jamais buffuriser les messages.
Résultats théoriques interessants, mais non utilisable dans les autres parties, car vraiment sépcifiques au fait qu'on ne buffurise rien.\\
Courbes de résultats
\subsection{Flux desynch avec un seul buffer dans le datacenter}
Probleme PALL, explication qu'on découpe le probleme en aller/retour.
Au retour algo FPT basé du simons adapté pour la periodicité, a l'aller on a regardé ce que différents ordres donnaient.\\
Courbes de résultats
\section{Flux synchronisés}
Ici, on utilise des buffers dans le réseau. On a pas besoin de distinguer les différentes topologies, car les algos les resolvent toutes. \\
Rappeler les notions de conflict depth. dire qu'ici pour 1 c'est trivial et que apres nos algos s'adaptent a toutes les conflict depth, même si en pratique il vaut mieux qu'elle soit petite.
\subsection{Compact form}
Reprendre la définition de compact assignment etc
\subsection{Algorithms}
On a commencé par regarder différents algos greedy, rien ne sortait du lot.
Expliquer les différents algos de voisinage, tout ce qu'on a regardé dessus (tabou avec ou sans mémoire, Descente d'un point aléatoire ou non, Recuit avec ou sans descente avant, réglage parametres recuit...\\
Parler des coupes pour de l'algo branch and bound.\\
Courbes de résultats
\section{Gestion d'un second type de flux, BE}
Rappeler qu'il est possible avec TSN de mixer deux types de flux en gérant le traffic et donc en priorisant les flux CRAN. Le but de cette section est de montrer l'impact sur le best effort.
\subsection{travaux n-GREEN}
Expliquer en quoi la technologie fait qu'il est trivial d'organiser les flux C-RAN entre eux pour qu'il n'y ait pas de contention. 
Dire que du coup on essaye de les organiser pour impacter au moins le best effort. Résultat, le best effort est même avantagé.
\subsection{BE dans les reseaux non optiques}
Dire que dans un cas plus général, on à pas ce changement de support qui rend l'organisation des flux facile. On peut néanmoins adapter le model en agrandissant la taille des paquets pour faire passer du best effort dedans. Regarder, cette derniere idée contre le fait de ne rien faire de particulier.
\section{Adaptation industrielle}
\subsection{Onos}
Présentation de ONOS, TSN et de FPGA. Déscription de ce qu'olivier à programmé sur ONOS, du switch FPGA de brice et de comment on connecte tout ça. 
\subsection{ANR N-GREEN}
Parler du prototype de dominique C. N-GREEN
\section{Ouverture, conclusion}

\subsection{Reinforcement learning}
\subsection{Model plus large avec tout ce qu'on a pas utilisé}
-Liens de vitesse différentes, tailles de paquets différents...
\subsection{questions ouvertes / suite de la these}

\end{document}
