\documentclass[a4paper,10pt]{article}
\usepackage[utf8]{inputenc}
\usepackage{graphicx,graphics} 
\usepackage{color}
\newcommand{\red}[1]{{\color{red} {#1}}}
\newcommand{\todo}[1]{{\color{red} TODO: {#1}}}
%opening

\title{Intégration Laboratoire Lineact}
\author{Maë Guiraud}
\graphicspath{{fig/}}
\begin{document}

\maketitle

\section{Profil:}

\textbf{Formation :}
\begin{itemize}
\item Master 2 AMIS (Algorithmes et Modélisation à l’Interface des Sciences) à l’UVSQ. 
\item Doctorat en informatique de l’université Paris Saclay (thèse CIFRE avec Nokia Bell Labs France).
\end{itemize}

\textbf{Sujet de thèse : Ordonnancements périodiques de messages pour minimiser la latence dans les réseaux dans un contexte 5G et au-delà.}
\begin{itemize}
\item Analyse de complexité.
\item Algorithmique.
\item Optimisation combinatoire.
\end{itemize}
J'ai étudié des problèmes d'ordonnancements périodiques pour lesquels je propose différentes approches algorithmiques dont l'utilisation des méta-heuristiques par voisinage.
\\

D'un point de vue thématique, mes travaux sont applicables aux cas d'applications nécessitant un \textbf{contrôle à distance et une virtualisation de fonctions} dans les systèmes en temps réels. Les principaux domaines d'application concernés sont :
\begin{itemize}
\item Industrie 4.0.
\item Contrôle aérien à distance.
\item Cloud-RAN.

\end{itemize}

Je suis actuellement ingénieur de recherche pour le laboratoire HYPHES, une collaboration entre le laboratoire DAVID (Données et Algorithmes pour une Ville Intelligente et Durable) et l’entreprise DCBrain qui travaille sur des sujets appliqués étroitement liés à la ville du futur :
Optimisation de réseaux de logistique (camions, taxis, bus...) Modélisation et optimisation de réseaux urbains et inter-urbains de gaz.

Je demanderais la qualification du CNU 27 lors de la campagne 2022.



\section{Projet d'intégration et de recherche}
\subsection{Domaines scientifiques de compétences :}
\begin{itemize}
\item Modélisation discrète et stochastique.
\item Étude théorique de problèmes et analyse de la complexité.
\item Intelligence computationnelle : algorithmes d'optimisations.
\item Travaux en cours (labo HYPHES) : méthodes d'optimisation et d'intelligence artificielles, techniques d'apprentissage par renforcement, théorie des jeux.
\end{itemize}
\subsection{Applications :}

L'industrie 4.0 (ou industrie du futur) consiste à re penser la production grâce aux outils numériques et dans le but d'améliorer le rendement, les conditions de travail des salariés ou de réduire l'impact environnemental des chaines de production. Les usines du futur sont aujourd'hui composées de nombreux capteurs permettant de collecter un grand nombre de données numériques   (température, pression, géolocalisation des pièces sur la chaine de production, problème mécanique sur une machine, etc...). Ces nombreux capteurs sont inter-connectés et forment un grand réseau dans lequel toutes les entités (machines, produits, employés) communiquent.
 Mainrenance prévisionelle, 
 
 production adaptée aux besoins
 
 big data-> intelligence artificielle
 
  cobots robots colaboratoifs: pilotés a distance, réduisent cadence quand un humain s'approche

 D'un coté, je pense qu'un des enjeux majeurs est de minimiser la latence de transmission afin de garantir une fiabilité dans le système de contrôle. D'un autre côté, l'approche centralisée de l'usine dans laquelle chaque machine serait asservie à un contrôleur central peut poser de vrais problèmes de complexité algorithmique pour des calculs en temps réels. Dans ce contexte, je propose de travailler sur une approche distribuée, dans laquelle chaque machine serait capable de s'adapter, selon ce qu'elle peut faire, à l'état du système.
Ce principe d'adaptation à l'état du système peut être vu à un autre niveau de distribution : je propose de travailler sur des approches permettant aux industries de s'adapter à un aléa intervenant ailleurs que dans la chaîne de production (comme une livraison retardée ou un manque de stockage, par exemple).\\
\newline
Cette approche d'un système distribué se retrouve dans une smart grid dans laquelle les différents acteurs peuvent être vus comme des entités intelligentes cherchant à optimiser leurs consommations ou bien la consommation globale de la smart grid. Je propose de regarder différentes approches afin de pouvoir configurer le réseau et de définir quel acteur sera producteur et quel acteur sera consommateur d'énergie en temps réel.\\
\newline
Les Bâtiments du futur, où les villes du futur proposent des "jumeaux numériques" (BIM/CIM). Ces Jumeaux numériques sont des outils permettant de modéliser au mieux ces infrastructures, dans lesquelles de nombreux problèmes d'allocation de ressources se posent. Pour le bâtiment du futur par exemple, j'envisage de travailler sur des solutions de maîtrise énergétiques avec un mixte entre optimisation et apprentissage, avec prédiction des conditions environnementales et des usages. Afin de mener à bien ces travaux, j'envisage de collaborer avec Benjamin Cohen Boulakia ainsi que l'équipe du laboratoire DAVID de l'UVSQ. Les villes du futur (CIM, TIM) peuvent aussi de mon point de vue être étudiées avec une approche distribuée. Par exemple, les flux de véhicules peuvent être gérés avec des politiques distribuées en chaque intersection, dans un modèle ou toutes les intersections viseraient à réduire la congestion du trafic dans la ville.


Je pense aussi pouvoir valoriser mes travaux de thèse visant à garantir une latence déterministe et minimale dans chacun des cas d'applications cités où les communications en temps réels sont cruciales. 


\section{Enseignements}
De part l'expérience que j'ai acquise, je pense être capable d'enseigner les modules liés à l'informatique théorique et à la programmation, ainsi que tous les enseignements sur les outils numériques liés aux formations proposées.

Voici la liste des enseignements que j'ai eu l'occasion de donner au cours de ma thèse :
\begin{itemize}
\item Structure de données et Algorithmes : licence 2 ($\simeq$ 60h) et Master 1 ($\simeq$ 30h) Algorithmes de graphes : licence 2 ($\simeq$ 30h)
\item Programmation (C,Java) : licence 1-2 (+100h)
\item Systèmes d’exploitations : pour la formation ISN des professeurs de maths de lycée.
\item Encadrement d’un étudiant de DUT en stage
\end{itemize}


    Projet d’intégration dans LINEACT CESI, d’une longueur d’environ deux pages, constitué de :

               -Thème et axe de recherche,

               -Expertise maitrisée à mobiliser, ou expertise visée si mobilité thématique,

               -Domaine d’application visé,

               -Question(s) de recherche soulevée(s),

               -EC LINEACT CESI avec lesquels le candidat souhaite collaborer en précisant son apport  
               
               
               Prévois environ 3 pages, décrivant :

·         Un sujet général (un peu plus large qu’un sujet de thèse, pense projet)

·         Les enjeux de ce sujet (scientifiques, technologiques, sociétaux)

·         Les compétences scientifiques nécessaires

o   dont tu disposes et qui seront mises en action

Et/ou

o   qui te manquent, pour lesquelles tu devras donc te former

·         Les perspectives de collaboration au sein de LINEACT (pas seulement Nanterre).

 

@+

\end{document}
