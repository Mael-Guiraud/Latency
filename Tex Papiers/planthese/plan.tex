\documentclass[a4paper,10pt]{article}
\usepackage[utf8]{inputenc}
\usepackage{graphicx,graphics} 
\usepackage{color}
\newcommand{\red}[1]{{\color{red} {#1}}}
\newcommand{\todo}[1]{{\color{red} TODO: {#1}}}
%opening

\title{Intégration Laboratoire Lineact}
\author{Maë Guiraudl}
\graphicspath{{fig/}}
\begin{document}
 
\maketitle

\section{Profil:}

\textbf{Formation :}
\begin{itemize}
\item Master 2 AMIS (Algorithmes et Modélisation à l’Interface des Sciences) à l’UVSQ. 
\item Doctorat en informatique de l’université Paris Saclay (thèse CIFRE avec Nokia Bell Labs France).
\end{itemize}

\textbf{Sujet de thèse : Ordonnancement périodiques de messages pour minimiser la latence dans les réseaux dans un contexte 5G et au delà.}
\begin{itemize}
\item Modélisation précise du problème pratique étudié.
\item Analyse de compléxité.
\item Développement de divers algorithmes d’ordonnancement périodiques:
\begin{itemize}
\item Algorithmes gloutons avec analyse théorique d’une garantie de résultat.
\item Algorithmes de résolution exacte des problèmes (FPT).
\item Utilisation des méta-heuristiques (Descente, recherche tabou, recuit simulé).
\end{itemize}
\end{itemize}


 
Domaines d’applications de mon sujet de thèse:
 \begin{itemize}
\item Industrie 4.0.
\item Tour de contrôle à distance.
\item Plus généralement, tout réseau dans lequel on veut limiter la latence au temps physique de transit.
 
\end{itemize}

        Je suis actuellement ingénieur de recherche pour le laboratoire HYPHES, une collaboration entre le laboratoire DAVID (Données et Algorithmes pour une Ville Intelligente et Durable) et l’entreprise DCBrain qui travaille sur des sujets appliqués étroitement liés à la ville du futur:
Optimisation de réseaux de logistique (camions, taxis, bus...) Modélisation et optimisation de réseaux urbains et inter-urbains de gaz.



\section{Projets de recherche}
\subsection{Domaines d'applications:}
J'envisage de travailler sur tout sujet ou des techniques d'optimisation algorithmiques peuvent être appliquées. Dans le contexte de la ville intelligente, plusieurs sujets m'intéressent:\begin{itemize}
\item Optimisation énergétique, a l'échelle d'une ville (smart grid) ou du bâtiment.
\item Les réseaux de transports (Logistique, déplacements humains ou matériels)
\item ... 
\end{itemize}

Je pense aussi pouvoir valoriser certains de mes travaux de thèse dans le cadre de l'industrie 4.0, où les communications en temps réelles sont cruciales. De plus ce domaine fait de plus en plus appel à des technique d'intelligence artificielles (Reinforcement learning, théorie des jeux, ...) sur lesquelles je trouverais très enrichissant de travailler.
\subsection{Domaines scientifiques de compétences:}
\begin{itemize}
\item Modélisation et analyse de la complexité d’un problème.
\item Plusieurs domaines de l’informatique théorique.
\item Maitrise des techniques d’optimisation: ordonnancement, méta-heuristiques.
\item Travaux en cours (labo HYPHES): Techniques de reinforcement learning, théorie des jeux.
\end{itemize}
\section{Enseignements}
\begin{itemize}
\item Structure de données et Algorithmes: Licence 2 ($\simeq$ 60h) et Master 1 ($\simeq$ 30h) Algorithmes de graph: Licence 2 ($\simeq$ 30h)
\item Programmation (C,Java): Licence 1-2 (+100h)
\item Sytème d’exploitations: Pour la formation ISN des professeurs de maths de lycée.
\item Encadrement d’un étudiant de DUT en stage
\end{itemize}
\end{document}
