\documentclass[a4paper,10pt]{article}
\usepackage[utf8]{inputenc}
\usepackage{graphicx,graphics} 
\usepackage{color}
\newcommand{\red}[1]{{\color{red} {#1}}}
\newcommand{\todo}[1]{{\color{red} TODO: {#1}}}
%opening

\title{Intégration Laboratoire Lineact}
\author{Maë Guiraudl}
\graphicspath{{fig/}}
\begin{document}
 
\maketitle

\section{Profil:}

\textbf{Formation :}
\begin{itemize}
\item Master 2 AMIS (Algorithmes et Modélisation à l’Interface des Sciences) à l’UVSQ. 
\item Doctorat en informatique de l’université Paris Saclay (thèse CIFRE avec Nokia Bell Labs France).
\end{itemize}

\textbf{Sujet de thèse : Ordonnancement périodiques de messages pour minimiser la latence dans les réseaux dans un contexte 5G et au delà.}
\begin{itemize}
\item Analyse de complexité.
\item Algorithmique.
\item Optimisation combinatoire.
\end{itemize}
 J'ai étudié des problèmes d'ordonnancement périodique pour lesquelles je propose différentes approches algorithmiques dont l'utilisation des méta-heuristiques par voisinage.
\\
 
D'un point de vue thématique, mes travaux sont applicables aux cas d'applications nécessitant un \textbf{contrôle à distance et une virtualisation de fonctions} dans les système en temps réels: 
 \begin{itemize}
\item Industrie 4.0.
\item Contrôle aérien à distance.
\item Cloud-RAN.
 
\end{itemize}

        Je suis actuellement ingénieur de recherche pour le laboratoire HYPHES, une collaboration entre le laboratoire DAVID (Données et Algorithmes pour une Ville Intelligente et Durable) et l’entreprise DCBrain qui travaille sur des sujets appliqués étroitement liés à la ville du futur:
Optimisation de réseaux de logistique (camions, taxis, bus...) Modélisation et optimisation de réseaux urbains et inter-urbains de gaz.



\section{Projet d'intégration et de recherche}
\subsection{Domaines scientifiques de compétences:}
\begin{itemize}
\item Modélisation discrete et stochastique.
\item Etude théorique de problèmes et analyse de la complexité.
\item Intelligence computationnelle : Algorithmes d'optimisations.
\item Travaux en cours (labo HYPHES): Méthodes d'optimisation et d'intelligence artificielles, techniques de reinforcement learning, théorie des jeux.
\end{itemize}
\subsection{Domaines d'applications:}

L'industrie 4.0 propose de vrais enjeux de latence et de fiabilité. Les usines sont dotés de nombreux systèmes inter-connectés. D'un coté, je pense qu'un des enjeux majeurs est de minimiser la latence de transmission afin de garantir une fiabilité dans le système de contrôle. D'un autre coté, l'approche centralisée de l'usine dans laquelle chaque machine serait asservie à un contrôleur central peut poser de vrais problèmes de complexité algorithmique pour des calculs en temps réels. Dans ce contexte, je propose de travailler sur une approche distribuée, dans laquelle chaque machine serait capable de s'adapter, selon ce qu'elle peut faire, à l'état du système.
Ce principe d'adaptation à l'état du système peut être vu à un autre niveau de distribution: Je propose de travailler sur des approches permettant aux industries de s'adapter à un aléa intervenant ailleurs que dans la chaine de production (comme une livraison retardée ou un manque de stockage, par exemple).\\
\newline
Cette approche d'un système distribué se retrouve dans une smart grid dans laquelle les différents acteurs peuvent être vus comme des entités intelligentes cherchant à optimiser leurs consommations ou bien la consommation globale de la smart grid. Je propose de regarder différentes approches afin de pouvoir configurer le réseau et de définir quel acteur sera producteur et quel acteur sera consommateur d'énergie en temps réel.\\
\newline
Les Bâtiments du futurs, où les villes du futures proposent des "jumeaux numériques" (BIM/CIM). Ces Jumeaux numériques sont des outils permettant de modéliser au mieux ces infrastructures, dans lesquelles de nombreux problèmes d'allocation de ressources se posent. Pour le bâtiment du futur, par exemple je proposer de travailler sur des solutions de maitrise énergétiques s'adaptant aux données relevées par les capteurs et aux facteurs extérieurs (température, usagers).



Je pense aussi pouvoir valoriser mes travaux de thèse visant à garantir une latence non déterministe et minimale dans chacun des cas d'applications cités où les communications en temps réelles sont cruciales. 

\section{Enseignements}
De part l'experience que j'ai acquise, je pense être capable d'enseigner les modules liés à l'informatique théorique et à la programmation, ainsi que tous les enseignements sur les outils numériques liés aux formations proposées.

Voici la liste des enseignements que j'ai eu l'occasion de donner au cours de ma thèse:
\begin{itemize}
\item Structure de données et Algorithmes: Licence 2 ($\simeq$ 60h) et Master 1 ($\simeq$ 30h) Algorithmes de graph: Licence 2 ($\simeq$ 30h)
\item Programmation (C,Java): Licence 1-2 (+100h)
\item Sytème d’exploitations: Pour la formation ISN des professeurs de maths de lycée.
\item Encadrement d’un étudiant de DUT en stage
\end{itemize}


\end{document}
