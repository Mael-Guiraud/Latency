\documentclass[a4paper,10pt]{article}
\usepackage[utf8]{inputenc}
\newcommand{\red}[1]{{\color{red} {#1}}}
%opening
\title{Plan de thèse}
\author{Maël}

\begin{document}

\maketitle

\begin{abstract}
Voici mon plan de thèse.
\end{abstract}

\section{Generalités}
\subsection{Introduction}
Introduction du contexte C-RAN, résumé des travaux effectués, des resultats.
\subsection{Notions utiles}
Rappel des notions souvent utilisées dans la thèse
\subsection{Model}
Définition du model avec exactement ce qu'on utilise,et des problemes (PALL, PAZL et SPALL), Npcomplétude des problemes
\subsection{Travaux similaires}
Expliquer ce qu'on a lu, pourquoi les models les plus proches sont différents, ou quelles sont les approches différentes que les gens ont eu.
\section{Flux désynchronisés}
Expliquer qu'on regarde l'étoile, parler des articles similaires
\subsection{Flux desynchronisés sans buffer}
Tout les travaux sur PAZL, notamment le papier algotel
\subsection{Flux desynch avec un seul buffer au debut de la route}
Travaux sur PALL dans l'étoile. Justifier que c'est cool dans les réseaux optiques ou on ne buffurise jamais les routes.
\section{Flux synchronisés}
Travaux actuels sur SPALL
\section{Gestion d'un second type de flux, BE}
\subsection{travaux n-GREEN}
\subsection{BE dans les reseaux non optiques}
\section{Adaptation industrielle}
Ce qu'on a fait avec ONOS
\section{ouverture, conclusion}
\subsection{Reinforcement learning}
\subsection{model avec tout ce qu'on a pas utilisé plus grand}
-Liens de vitesse différentes, tailles de paquets différents...


\end{document}
