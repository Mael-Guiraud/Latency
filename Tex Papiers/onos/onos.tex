%\documentclass[a4paper,10pt]{article}
\documentclass[10pt]{article}
\usepackage[utf8]{inputenc}
\usepackage{xspace}
\usepackage{url}
\usepackage{graphicx,graphics} 
\usepackage{color}
\usepackage{amsmath}
\usepackage{amsfonts}
\usepackage{amssymb}
\usepackage{amsthm}
\usepackage{algorithm}
\usepackage{algorithmic}
\usepackage{longtable}
\usepackage{complexity}
\usepackage{tkz-graph}
\usepackage{float}
\usepackage{tabularx}
\usepackage{setspace}
\usepackage{icomma}
\renewcommand{\algorithmicrequire}{\textbf{Input:}}
\renewcommand{\algorithmicensure}{\textbf{Output:}}
\usepackage{authblk}
\usepackage[colorlinks=true,breaklinks=true,linkcolor=blue]{hyperref}


% \renewcommand{\thefootnote}{\*}

\newcommand{\todo}[1]{{\color{red} TODO: {#1}}}
\newcommand\pazl{\textsc{pazl}\xspace}
\newcommand\pall{\textsc{pall}\xspace}
\newcommand\bra{\textsc{bra}\xspace}
\newcommand\pra{\textsc{pra}\xspace}
\newcommand\minpra{\textsc{min-pra}\xspace}
%opening
\title{Onos}
 


\author[1,2]{Ma\"el Guiraud}

\affil[1]{David Laboratory, UVSQ}
\affil[2]{Nokia Bell Labs France}

\begin{document}
\maketitle

ONOS is an open source network OS, that allows providers to manage an SDN network. Once it is launched, the ONOS software is a controller and can communicate with the network equipment connected to it. Several applications have already been developed and allows to monitor the network though ONOS. Our goal is to understand the mechanisms of an ONOS application in order to develop our own application that could :
\begin{itemize}
\item Communicate with the network equipments (with the OpenFlow protocol, for instance) in order to collect the informations on the network  topology, etc, and to send the informations needed by the equipments in our time sensitive communications.

\item Communicate with our software that computes the specific solution for the network, and send the scheduling to the ONOS application.
\end{itemize}

\end{document}
