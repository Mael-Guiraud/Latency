%%%%%%%%%%%%%%%%%%%%%%%%%%%%%%%%%%%%%%%%%%%%%%%%%%%%%%%%%%%%%%%%%%%%%
%%                                                                 %%
%% Please do not use \input{...} to include other tex files.       %%
%% Submit your LaTeX manuscript as one .tex document.              %%
%%                                                                 %%
%% All additional figures and files should be attached             %%
%% separately and not embedded in the \TeX\ document itself.       %%
%%                                                                 %%
%%%%%%%%%%%%%%%%%%%%%%%%%%%%%%%%%%%%%%%%%%%%%%%%%%%%%%%%%%%%%%%%%%%%%

%%\documentclass[referee,sn-basic]{sn-jnl}% referee option is meant for double line spacing

%%=======================================================%%
%% to print line numbers in the margin use lineno option %%
%%=======================================================%%

%%\documentclass[lineno,sn-basic]{sn-jnl}% Basic Springer Nature Reference Style/Chemistry Reference Style

%%======================================================%%
%% to compile with pdflatex/xelatex use pdflatex option %%
%%======================================================%%

%%\documentclass[pdflatex,sn-basic]{sn-jnl}% Basic Springer Nature Reference Style/Chemistry Reference Style

%%\documentclass[sn-basic]{sn-jnl}% Basic Springer Nature Reference Style/Chemistry Reference Style
\documentclass[sn-mathphys]{sn-jnl}% Math and Physical Sciences Reference Style
%%\documentclass[sn-aps]{sn-jnl}% American Physical Society (APS) Reference Style
%%\documentclass[sn-vancouver]{sn-jnl}% Vancouver Reference Style
%%\documentclass[sn-apa]{sn-jnl}% APA Reference Style
%%\documentclass[sn-chicago]{sn-jnl}% Chicago-based Humanities Reference Style
%%\documentclass[sn-standardnature]{sn-jnl}% Standard Nature Portfolio Reference Style
%%\documentclass[default]{sn-jnl}% Default
%%\documentclass[default,iicol]{sn-jnl}% Default with double column layout

%%%% Standard Packages
%%<additional latex packages if required can be included here>
%%%%

%%%%%=============================================================================%%%%
%%%%  Remarks: This template is provided to aid authors with the preparation
%%%%  of original research articles intended for submission to journals published 
%%%%  by Springer Nature. The guidance has been prepared in partnership with 
%%%%  production teams to conform to Springer Nature technical requirements. 
%%%%  Editorial and presentation requirements differ among journal portfolios and 
%%%%  research disciplines. You may find sections in this template are irrelevant 
%%%%  to your work and are empowered to omit any such section if allowed by the 
%%%%  journal you intend to submit to. The submission guidelines and policies 
%%%%  of the journal take precedence. A detailed User Manual is available in the 
%%%%  template package for technical guidance.
%%%%%=============================================================================%%%%

\jyear{2023}%

%% as per the requirement new theorem styles can be included as shown below
\theoremstyle{thmstyleone}%
\newtheorem{theorem}{Theorem}%  meant for continuous numbers
%%\newtheorem{theorem}{Theorem}[section]% meant for sectionwise numbers
%% optional argument [theorem] produces theorem numbering sequence instead of independent numbers for Proposition
\newtheorem{proposition}[theorem]{Proposition}% 
%%\newtheorem{proposition}{Proposition}% to get separate numbers for theorem and proposition etc.

\theoremstyle{thmstyletwo}%
\newtheorem{example}{Example}%
\newtheorem{remark}{Remark}%

\theoremstyle{thmstylethree}%
\newtheorem{definition}{Definition}%

\raggedbottom
%%\unnumbered% uncomment this for unnumbered level heads

\newcommand\shortestlongest{\texttt{ShortestLongest}\xspace}
\newcommand\metaoffset{\texttt{MetaOffset}\xspace}
\newcommand\ESCA{\texttt{ESCA}\xspace}
\newcommand\greedydeadline{\texttt{GreedyDeadline}\xspace}
\newcommand\MLS{\texttt{MLS}\xspace}
\newcommand\PMLS{\texttt{PMLS}\xspace}
\newcommand\ASPMLS{\texttt{ASPMLS}\xspace}

\newcommand\SPMLS{\texttt{SPMLS}\xspace}
\newcommand\FIFO{\texttt{FIFO}\xspace}
\newcommand\framepre{\texttt{FramePreemption}\xspace}
\newcommand\critdead{\texttt{CriticalDeadline}\xspace}




% \renewcommand{\thefootnote}{\*}

\newcommand{\todo}[1]{{\color{red} TODO: {#1}}}
\newcommand\pazl{\textsc{pazl}\xspace}
\newcommand\pall{\textsc{pall}\xspace}
\newcommand\wta{\textsc{wta}\xspace}
\newcommand\pra{\textsc{pra}\xspace}
\newcommand\minpazl{\textsc{minpazl}\xspace}
\newcommand\mintra{\textsc{mintra}\xspace}

\begin{document}

\title[Deterministic Scheduling of Periodic Messages for Low Latency in Cloud RAN]{Deterministic Scheduling of Periodic Messages for Low Latency in Cloud RAN}

%%=============================================================%%
%% Prefix	-> \pfx{Dr}
%% GivenName	-> \fnm{Joergen W.}
%% Particle	-> \spfx{van der} -> surname prefix
%% FamilyName	-> \sur{Ploeg}
%% Suffix	-> \sfx{IV}
%% NatureName	-> \tanm{Poet Laureate} -> Title after name
%% Degrees	-> \dgr{MSc, PhD}
%% \author*[1,2]{\pfx{Dr} \fnm{Joergen W.} \spfx{van der} \sur{Ploeg} \sfx{IV} \tanm{Poet Laureate} 
%%                 \dgr{MSc, PhD}}\email{iauthor@gmail.com}
%%=============================================================%%

\author[1]{\fnm{Dominique} \sur{Barth}}\email{dominique.barth@uvsq.fr}

\author*[2]{\fnm{Maël} \sur{Guiraud}}\email{mguiraud@cesi.fr}

\author[1]{\fnm{Yann} \sur{Strozecki}}\email{yann.strozecki@uvsq.fr}

\affil[1]{\orgdiv{DAVID laboratory}, \orgname{Université de Versailles Saint-Quentin}, \orgaddress{\street{45 Av. des Etats Unis}, \city{Versailles}, \postcode{78000}, \country{France}}}

\affil[2]{\orgdiv{LINEACT}, \orgname{CESI}, \orgaddress{\street{93 Bd de la Seine}, \city{Nanterre}, \postcode{92000},  \country{France}}}



%%==================================%%
%% sample for unstructured abstract %%
%%==================================%%

\begin{abstract}
Cloud-RAN (C-RAN) is an architecture for cellular networks, where processing units, previously attached to antennas, are centralized in data centers. The main challenge, to fulfill protocol time constraints, is to minimize the latency of the periodic messages sent from the antennas to their processing units and back. We show that statistical multiplexing suffers from high logical latency, due to buffering at nodes to avoid collisions. Hence, we propose to use a \emph{deterministic} scheme for sending periodic messages \emph{without collision} in the network, thus saving the latency incurred by buffering.

 
We propose several algorithms to compute such schemes for star routed networks, a common topology where one link is shared by all antennas. First, we show there exist deterministic sending schemes without any buffering when the routes are short or the load is small. When the load is high, we allow buffering in processing units, and we propose the \PMLS algorithm adapted from a classical scheduling method. Experimental results show that, even under full load, \PMLS finds a deterministic sending scheme with no logical latency most of the time, while using statistical multiplexing adds very large latency. Moreover, \PMLS runs in polynomial time and scales well to hundreds of antennas. Building on this algorithm, we also obtain very low latency periodic sending schemes that do not disrupt additional random traffic on the network. This article is an extended version of previous work presented at ICT~\cite{Guir1806:Deterministic}.
\end{abstract}
\keywords{Deterministic networking, Time-sensitive networking, Periodic scheduling, Low latency, Zero jitters, C-RAN}

%%\pacs[JEL Classification]{D8, H51}

%%\pacs[MSC Classification]{35A01, 65L10, 65L12, 65L20, 65L70}

\maketitle

\section*{Declarations}
Not applicable



%%===========================================================================================%%
%% If you are submitting to one of the Nature Portfolio journals, using the eJP submission   %%
%% system, please include the references within the manuscript file itself. You may do this  %%
%% by copying the reference list from your .bbl file, paste it into the main manuscript .tex %%
%% file, and delete the associated \verb+\bibliography+ commands.                            %%
%%===========================================================================================%%
\bibliography{sn-bibliography}% common bib file
%% if required, the content of .bbl file can be included here once bbl is generated
%%\input sn-article.bbl

%% Default %%
%%\input sn-sample-bib.tex%

\end{document}
