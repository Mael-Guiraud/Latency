%\documentclass[a4paper,10pt]{article}
\documentclass[10pt]{article}
\usepackage[utf8]{inputenc}
\usepackage{xspace}
\usepackage{url}
\usepackage{graphicx,graphics} 
\usepackage{color}
\usepackage{amsmath}
\usepackage{amsfonts}
\usepackage{amssymb}
\usepackage{amsthm}
\usepackage{algorithm}
\usepackage{algorithmic}
\usepackage{longtable}
\usepackage{complexity}
\usepackage{tkz-graph}
\usepackage{float}
\usepackage{tabularx}
\usepackage{setspace}
\usepackage{icomma}
\renewcommand{\algorithmicrequire}{\textbf{Input:}}
\renewcommand{\algorithmicensure}{\textbf{Output:}}
\usepackage{authblk}
\usepackage[colorlinks=true,breaklinks=true,linkcolor=blue]{hyperref}


\newcommand\rmatching{${\cal R}$-matching\xspace}
\newcommand\mdelay{$\cal M$-delay\xspace}
\newcommand\matchedgraph{{\bf matched graph}}
\newtheorem{proposition}{Proposition}
\newtheorem{theorem}{Theorem}

\setlength{\parskip}{1ex} % Espace entre les paragraphes

\newtheorem{fact}{Fact}
\newtheorem{lemma}[theorem]{Lemma}
\newtheorem{definition}{Definition}
\newtheorem{corollary}{Corollary}

% \renewcommand{\thefootnote}{\*}

\newcommand{\todo}[1]{{\color{red} TODO: {#1}}}
\newcommand\pazl{\textsc{pazl}\xspace}
\newcommand\pall{\textsc{pall}\xspace}
\newcommand\bra{\textsc{bra}\xspace}
\newcommand\pra{\textsc{pra}\xspace}
\newcommand\minpra{\textsc{min-pra}\xspace}
%opening
\title{Deterministic Scheduling of Periodic Messages for Cloud RAN}
 

\author[1]{Dominique Barth}
\author[1,2]{Ma\"el Guiraud}
% \author[1]{Christian Cad\'er\'e}
 \author[2]{Brice Leclerc}
 \author[2]{Olivier Marc\'e}
\author[1]{Yann Strozecki}
\affil[1]{David Laboratory, UVSQ}
\affil[2]{Nokia Bell Labs France}

\begin{document}

The {\bf physical delay} of a route is the time taken by a message to travel the links of the route. It corresponds to the sum on all the weight of the arcs of the route. \cite{guillemin_peak_1992,demichelis_ip_nodate} 

In our study, when a datagram is sent, some {\bf physical buffers } and {\bf logical buffers} can increase the physical delay of the route. The {\bf delay} of a datagram on a route is then equal to the sum of the physical delay of the route, and the physical and logical buffers.

As a reminder, a stream is a sequence of datagram. Each period, the same sequence of datagrams is sent on a route, in the same order. Let us call $d$ a datagram of a stream sent during the first period, and $d'$ the same datagram, sent in the next period.
The {\bf packet delay variation} is the variation of the delay between $d$ and $d'$. In our study, since the logical delays are deterministic, computed on a period and then applied to all the next periods, the packet delay variation is equal to zero for each datagrams.
Also, the {\bf jitter}, which correspond to the difference between the minimal and maximal packet delay variation of a datagram is equal to zero.

\bibliographystyle{ieeetr}
\bibliography{srcs}

\end{document}
