\documentclass[english]{article}
\usepackage[T1]{fontenc}
\usepackage[utf8]{inputenc}

\usepackage{babel}
\usepackage{authblk}
\usepackage{xspace}
\usepackage{graphicx,graphics} 
\usepackage{color}
\usepackage{amsmath}
\usepackage{amsfonts}
\usepackage{amssymb}
\usepackage{amsthm}
\usepackage{algorithm}
\usepackage{algorithmic}
\usepackage{longtable}
\usepackage{complexity}
\usepackage{hyperref}
\usepackage{tkz-graph}
\begin{document}

\title{Scheduling algorithm to avoid contention in meshed networks}
 

\newcommand{\todo}[1]{{\color{red} TODO: {#1}}}
\newcommand\spall{\textsc{SPALL}\xspace}
\newtheorem{theorem}{Theorem}
\newtheorem{lemma}[theorem]{Lemma}
\newtheorem{definition}{Definition}
\newtheorem{corollary}{Corollary}
\author[1]{\bf{ {Dominique Barth}}}
\author[1,2]{\bf{ {Ma\"el Guiraud}}}
\author[2]{\bf{ {Brice Leclerc}}}
\author[2]{\bf{ {Olivier Marc\'e}}}
\author[1]{\bf{ {Yann Strozecki}}}


\affil[1]{David Laboratory, UVSQ}
\affil[2]{Nokia Bell Labs France}


\maketitle

\section*{Introduction}
\section{Model \& Problems}
\subsection{Model}

We use the notation $[n]$ to denote the interval of $n$ integers $\{0,\dots,n-1\}$.

  \subsubsection{Discrete time model}
  In the model presented here, the time is discrete. The unit of time  is called a {\bf tic}. This is the time needed to transmit an atomic data over a link of the network. We consider that the speed of the links is the same over all the links of the network. 
In this paper, the size of an atomic data is 64B, the speed of the links is considered to be 10Gbps, \todo{rajouter durée d'un tic}.
  \subsubsection{Network modeling}
  We study a communication network in which some pairs of source-destination nodes between which some messages are sent. Also, the routing between each such a pair of nodes is given.
The network is modeled as a directed acyclic multigraph $G=(V,A)$. The set of vertices is composed of three non intersecting subsets: a subset $\cal{S}$ of vertices modeling the sources of the messages, a subset $\cal{D}$ modeling the destination of the messages, and a subset $\cal{C}$ representing the contention points. Indeed, some links of the network are shared between several pairs of nodes source-destination, we represent the begining of this link (i.e. the pyhsical node which send the messages into this link) as a contention point. Each arc  $(u,v)$ in $A$ is labeled by an integer weight $\Omega(u,v)$ which represents the number of tics elapsed between the sending tic of the message in $u$ and the reception tic of this message in $v$ using this arc. An arc in $G$ can represent several physical links or nodes, which do not induce contention points.\\
  A {\bf route} $r$ in $G$ is a directed path, that is, a sequence of adjacent vertices $u_1, \ldots , u_{l}$, with $(u_i,u_{i+1}) \in A$.  The {\bf weight of a vertex} $u_i$ in a route $r=(u_1,\dots,u_l)$ is defined by $\lambda(u_i,r)= \sum\limits_{1 \leq j <i} \Omega(u_j, u_{j+1})$. It is the number of tics needed between the sending tic of a message in the first vertex of the route and the reception tic of this message at $u_i$. We also define $\lambda(u_1,r)=0$. The weight of the route $r$ is defined by $\lambda (r)= \lambda (u_l,r)$.
We denote by $\cal R$ a set of routes, the pair $(G,\cal R)$ is called a {\bf routed network} and represents our telecommunication network.\\
The \textbf{contention level} of a node $u_i \in \cal{C}$ on a route $r$ is defined by $cl(u_i,r) = i-1$. This is the id of contention point on the route. The contention level of a node $u$ is $cl(u) = \max_{1\leq i\leq n} cl(u,i)$, with $n$ the number of routes using the node $u$.
\todo{petit dessin avec un reseau et sa représentation en graph}
The conflict depth of a routed network $(G,\cal{R})$ is equal to the maximum of the contention level of its vertices.
 \subsection{Messages dynamic}
	     
        In the process we study, a {\bf datagrams} is sent on each route from the source node. The size $|d|$ of a datagram $d$ is an integer, and correspond to the number of tics needed by a node $u$ to emit the datagram through a link. Once a datagram has been emitted, it cannot be fragmented during its travel in the network. We consider that the datagrams sent on each routes are all available to be sent at the same tic, which can be simplified, wlog. by $0$.
      
        
          Let $r=(u_0,\dots,u_l)$ be a route. In order to avoid the contention, one can choose to buffer a datagram $d_r$ in any vertex $u_i$ of the route. The function $b(d_r,u_i), i \in \{0,\ldots,l\}$ associate to each couple datagram-vertex of the route an integer. Those integers represent the buffering time of the datagram in each nodes of the route.\\
       

 An \textbf{assingment} $a(u)$ of a vertex $u$ is a choice of a buffer value for each route using this vertex. The assingment $A(G,\cal{R})$ of a graph is defined by $\cup_{u \in V} a(u)$. In order to siplify the reading, we will note the assignment of a graph $A$, since the routed network $(G,\cal{R})$ is implicit from the context. The \textbf{reception date} of a datagram $d_r$ in vertex $u_i$ in $r$, is the first tic at which the datagram sent on $r$ reaches $u_i$, and is defined by $t(d_r,u_i) = \lambda(u_i,r) + \sum_{k=0}^{i-1} b(d_r,u_k) $. In other words, the date at which a datagram reach a vertex $u_i$ correspond to the physical delay of the links, plus the buffers determined by the logical solution proposed.\\
 \textbf{sending date} of a datagram $d_r$ by vertex $u_i$ in $r$, is the first tic at which the datagram is sent by $u_i$. It is defined by $s(d_r,u_i) = t(d_r,u_i) +  b(d_r,u_i) $. This is the reception date of the datgram plus the buffer defined by the assignment for the datagram.\\
 
  Let $v_i$ be the last vertex of the route $i$.
       We define the \textbf{full transmission time} of an assingment $A$ as $TR(A) = \displaystyle \max_{i \in \{0,\ldots,n\}} t(d_i,v_i) $, where $n$ is the number of routes in the routed network. This is the time ellapsed before the reception of the first tic of the last datagram.
         In the application we study the full transmission time is bounded . We define $Tmax$ such that $TR(A) \leq Tmax$.
         
  \subsection{Cloud-RAN context}
     \todo{completement ré ecrire cette section en disant qu'on peut ou non décomposer la BBU, et que le graph est symetrique car c'est full duplex mais que ce n'est pas obligé}
     

     
  \subsection{Periodic sending of the datagrams}
 The process we describe here is \textbf{periodic} during each period of $P$ slots, a datagram is sent from each source node in the network at the same tic in the period.
             We denote by $\tau$ the number 
      of \emph{consecutive slots} necessary to transmit a message. In this paper, we assume that $\tau$ is the same for all routes. Indeed, the datagram sent by a source node into the network is the same, regardless of the route. Let us call $[t(d_r,u)]_{P,\tau}$ the set of time slots used a datagram on the by route $r$ at vertex $u$ in a period $P$, that is $[t(d_r,u)]_{P,\tau} = \{t(d_r,u) + i \mod P \mid 0 \leq i < \tau \}$. 

      Let $r_1$ and $r_2$ be two routes, on which datagrams are sent at time $\theta(d_1)$ and $\theta(d_2)$ in their first vertex $u_1$ and $u_2$ respectively.
      We say that the two routes have a {\bf collision} if they share an arc $(u,v)$ and $[t(d_1,u_1)]_{P,\tau} \cap [t(d_2,u_2)]_{P,\tau} \neq \emptyset$.\\

        A {\bf valid assignment} $A$ of a routed network $(G,\cal R)$ is an assignment in which \emph{no pair of routes has a collision}.
        
\subsection{Problems}
       
       \noindent {\bf Synchronized Periodic Assignment for Low Latency (\spall)} 

      \noindent {\bf Input:}  A symmetric routed network $(G,{\cal R})$, the period $P$, $Tmax$.
      
      \noindent {\bf Question:} does there exist a valid assignment $A$ of $(G,{\cal R})$ such that $ TR(A) \leq Tmax$ ?

      \noindent {\bf Optimisation problem:} Minimizing $TR(A)$.
    
     
    
\section{Compact representation of an assignment}





We explain in this section how to represent an assignment for the problem 
\spall over a vertex in a compact way. Then, this representation
can be used at any contention point in an instance of larger depth.
Not all solutions can be represented in a compact way, but there
is at least one optimal assignment for \spall wich is.
It allows to design FPT algorithm by going through all compact 
representations. We obtain good heuristic algorithms using taboo search or simulated annealing, since one can easily define the neighborood of a compact representation.\\



\begin{definition}
A compact representation of an assignment over $n$ routes is a pair $(O,S)$,
where $O$ is an order on the $n$ routes and $S$ is a subset of the $n$ routes.
\end{definition}

We now explain how to decompress a compact representation into an assignment.
Because we only consider one contention point $u$, we simplify the notation of the functions $b(d_i,u)$, $r(d_i,u)$ and, $s(d_i,u)$ by respectively $b(d_i)$, $r(d_i)$ and $s(d_i)$.
As a reminder those functions gives respectively the buffer, the reception date and the sendinc date of the datagram $d_i$ in the contention point $u$. Also, an assignment on one arc is defined as $a = (b(d_1),\dots,b(d_n))$.
%\todo{Je ne connais pas les notations pour \spall, je suppose qu'on a des temps
%d'arrivée sur l'arc $a_1,\dots,a_n$ et qu'on fixe des temps de départ sur l'arc 
%$d_1,\dots,d_n$. Je note un assignement A. Je remarque qu'on a souvent omis des 
%fonctions comme PT, TR la dépendence en l'assignement, du coup certaines choses sont 
%plus dures à écrire proprement}
Say w.l.o.g. that the datagrams are indexed in the order given by $O$.
We fix the sending date of the first datagram $s(d_1) = r(d_1)$, that is $b(d_1) = 0$, the datagram does not wait in a buffer. Then, in each period beginning by the first datagram, the datagrams will be in order. To simplify, we assume that $s(d_1) = r(d_1) = 0$, which can be obtained by removing $r(d_1)$ to all arrival times. We fix the sending date of the datagram in order, when the first $i$ datagrams have their sending date computed, we fix $s(d_{i+1})$ in the following way. 

If $i+1 \notin S$, then $s(d_{i+1}) >= r(d_{i+1})$ otherwise the datagram $i+1$ should go in the period after the one it is available in, that is $s(d_{i+1}) > (r(d_{i+1})/P + 1)P$.
The value of  $s(d_{i+1})$ is the smallest value which satisfies the previous constraint,
ensures that there are no collision with the first $i$ datagrams and  satisfies the order, that is $s(d_{i+1}) \mod P > s(d_i) \mod P$. 
It is possible that the process fails to find a correct value for $s(d_i)$ at some point,
in that case there are no assignment associated to this compact representation.

We denote this transformation by $Sol$, that is $Sol(O,S)$ is the solution previously defined
(the routed network is implicit) or a special value to denote there is no assignment compatible with this compact representation. 

We can also define an inverse function which from most assignment $a$ computes a compact representation, that we denote by $Compact(a)$. The function is defined only for
assignments $a = (b(d_1),\dots,b(d_n))$, such that the first datagram does not wait in a buffer, that is $a_1 = d_1$. Assume w.l.o.g that $r(d_1) = s(d_1) = 0$, by considering the equivalent problem where $r(d_i)= r(d_i) -r(d_1)$ and $s(d_i) = s(d_i) - r(d_1)$.
Compute the values $(s(d_i))\mod P$ and  call their order $O$. Let 
 $S$ be the set of $i$ such that $(s(d_i) \geq (r(d_i) / P + 1) P$. We let $Compact(a) = (O,S)$.


A compact representation of a solution for an instance of depth larger than $k$
is a list of compact representations, one for each contention arc. The following theorem explains why it is enough to explore the compact representations to solve \spall.

\begin{theorem}
Among all assignments $A$ for a routed network $G,{\cal R})$, there is a compact representation which minimizes $TR(A)$.
\label{theorem:compact}
\end{theorem}
\begin{proof}
Consider an assingment of a vertex $u$ $a(u) = (b(d_1),\dots,b(d_n))$. If $\forall i \in \{1,\ldots,n\}$ $b(d_i) > 0 $, then $s(d_i) > r(d_i)$. Then it is possible to compute the assignment $a(u)' = (b'(d_1),\dots,b'(d_n))$ in which $b'(d_i) = b(d_i) - \min\limits_{ 1\leq j \leq n} b(d_j)$. Consider a vertex $v$ such that $cl(v) = cl(u) +1$, or such that $v \in \cal{D}$ that is, $v$ comes after $u$ in any routes shared by both vertices.\\
For all routes $i$ passing through $v$, by definition, and since $b'(d_i,u) \leq b(d_i,u)$: $$t'(d_i,v) = \lambda(v,i) + \sum_{u \in i, cl(u,i) < cl(v,i)} b'(d_i,u)  \leq t(d_i,v)$$\\
Note that by reducing the buffers in a node $u_i$, we allow a datagram to reach the node $u_{i+1}$ earlier. Nevertheless, we do not change the order of the datagrams in $v$, even if it is possible.
By induction, if $v$ is the last vertex of the route, then : $$t'(d_i,v) \leq t(d_i,v) \Leftrightarrow TR(A') \leq TR(A)$$
Thus, for all existing assignments $A$ solving \spall, it is possible to find an equivalent assignment $A$ which have compact representation $Compact(A')$ and such that $TR(A') \leq TR(A)$.\\
\end{proof}


We want to find the assignment for which $TR(A)$ is minimal. To do so, as explained in theorem~\ref{theorem:compact}, we search between all the compact representations the one which minimize $TR(A)$.

\begin{lemma}
 The number of compact representation $(O,S)$ for a contention point with $k$ routes is $k!2^{k-1}$.
 \label{lemma:numberarcs}
\end{lemma}
\begin{proof}
Consider the datagrams are numeroted $d_1,\ldots,d_k$.
 There is $k$ routes on the contention point. Thus, there is $k!$ possible orders for the sequence of datagram in a period.
 Once the order $O$ of the compact representation is given, one can fix, w.l.o.g. $b(d_1) = 0$.
 Then, there is $k-1$ remaining datagram that can be set in $S$.
 This mean there is $k!2^{k-1}$ pairs of different $(O,S)$, i.e. compact representation for this arc.
\end{proof}

From lemma~\ref{lemma:numberarcs}, one can determine the numbre of compact representation of an entire router network.
We define $k_1,\ldots,k_l$ the number of routes on the $l$ contention points of a routed network.
\begin{theorem}
 The number of compact representation for a routed network $(G,\cal{R})$ is $\prod_{i=1}^l k_i!2^{k_i-1}$ .
 \label{theorem:numbergraph}
\end{theorem}


\begin{corollary}
 Problem \spall is FPT when parametrized by the number of routes and can be solved in $O(\prod_{i=1}^n n!2^{n})$, where $n$ is the number of routes on the graph.
\end{corollary}



\section{Find the best assignment}
\subsection{Greedy}
Greedy sans buffers dans le graph -> greedy avec buffers dans le graph.

\subsection{Local Search}

There is a large number of compact representation and to find the one which minimize $TR(A)$, we try serveral local search technique like ill climbing, Simulated annealing or Tabu search.\\

As a reminder, a compact representation is, for each contention arc of the graph, a pair $(O,S)$ of an order $O$ an a subset $S$ of the routes of the arc.
We define the neighbors of a compact represation as follow: for each routes of the graph, and for each contention arcs to which this route belongs, we try to swap the route with the following route in the order $O$ of the arc, and to put or not the route in the subset $S$ of the arc. Thus, if a route $i$ belongs to $k$ contention arcs, there is $2^{k+1}$ neighbors considering this route. Consider that the routes are numeroted $1,\ldots,n$ and belongs to $k_1,\ldots,k_n$ contention arcs, a compact representation has $\sum_{i=1}^n 2^{k_i +1} $ 


 
\subsection{FPT}
We define a \textbf{partial compact represetation} $Pcr$ of a routed network $(G,R)$ as a choice of pairs $(O,S)$ on a subset of the arcs of the graph.
The function $Bound(Pcr)$ gives a lower bound of $TR(G,{\cal R})$, considering Pcr.


 As explained on theroem~\ref{theorem:compact} there is a compact representation which minimizes $TR(G,{\cal R})$. We browse the entire set of compact representation to find the one which minimizes $TR(G,{\cal R})$.\\
 To do so, we design the following branch and bound algorithm. First, we determine an upper bound for our branch and bound algorithm by computing $TR(G,{\cal R})$ with a greed algorithm. Then, for all arcs (sorted by contention level), we enumerate all the possible compact representations. For each of those compact representation, we compute $Bound(Pcr)$. If the result given by this function is higher than the upper bound already found, we do no consider this compact representation. Otherwise, we go to the next arc. Once we have chose an pair $(O,S)$ for all arcs, we compute $Sol(O,S)$ for all arcs, and if the given $TR(G,{\cal R})$ is lower than the previous upper bound we update this value for the rest of the search.\\


In order to speed up the search, we made the following cuts:

définition des différentes coupes.\\
\section{Results}




\subsection{Génération des instances}
Dans le cas d'application C-RAN, on prends un nombre de routes limitées (<20) et des tailles de routes petites. Néanmoins, on regarde des instances pour lesquelles le problème est difficile.\

Définition du type de graph regardés, expliquer que les instances sont plus dures quand les routes sont de tailles similaires et quand on augmente le nombre de points de contentions.\\
\
\subsection{Détails d'implémentations des algorithmes de voisinage}
Taboo : Avec une hash table pour la mémoire, ce qui rend long la recherche quand on fait beaucoup de pas\\
Recuit, expliquer comment j'ai réglé la température etc\\

FPT ?

\section{Conclusion}
\bibliographystyle{ieeetr}
\bibliography{srcs}
\end{document}
