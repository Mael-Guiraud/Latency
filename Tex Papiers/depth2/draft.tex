\documentclass{article}
\usepackage[T1]{fontenc}
\usepackage[utf8]{inputenc}

\usepackage[french]{babel}
\usepackage{authblk}
\usepackage{xspace}
\usepackage{graphicx,graphics} 
\usepackage{color}
\usepackage{amsmath}
\usepackage{amsfonts}
\usepackage{amssymb}
\usepackage{amsthm}
\usepackage{algorithm}
\usepackage{algorithmic}
\usepackage{longtable}
\usepackage{complexity}
\usepackage{hyperref}
\usepackage{tkz-graph}
\begin{document}

\title{Scheduling algorithm to avoid contention in meshed networks}
 

\newcommand{\todo}[1]{{\color{red} TODO: {#1}}}


\author[1]{\bf{ {Dominique Barth}}}
\author[1,2]{\bf{ {Ma\"el Guiraud}}}
\author[2]{\bf{ {Brice Leclerc}}}
\author[2]{\bf{ {Olivier Marc\'e}}}
\author[1]{\bf{ {Yann Strozecki}}}


\affil[1]{David Laboratory, UVSQ}
\affil[2]{Nokia Bell Labs France}


\maketitle

\section*{Introduction}
\section{Model \& Problems}
\subsection{Model}
Presenter le model: 
A rajouter par rapport a l'ancien\begin{itemize}
\item la notion de buffers intermédiaires
\item la métrique, on compte le temps différemment 
\end{itemize}
A simplifier \begin{itemize}
\item Le debit sur les flux, qu'on considère tous les mêmes.
\item La notion de jitter, on ne s'y intéresse pas vraiment, mais c'est un point important pour justifier, peut être à remonter dans l'intro en le redéfinissant pour un réseau déterministe?
\end{itemize} 
\subsection{Problems}
Je propose de ne parler que de SPALL, nos résultats sur PALL conflict depth 2 sont très peu interessants

\section{Topologies}
Je propose de parler rapidement des deux topologies. On peut peut être faire une sous section parlant de l'étoile en expliquant pourquoi on à un bon algo fpt(en renvoyant sur l'autre papier) vu que ce n'est pas très long.
\section{Algorithms}
\subsection{Greedy}
Presenter les différents algo greedy, expliquer pourquoi on les essaye
\subsection{Local Search}
Differentes techniques de voisinages utilisées Notion de voisinage n1
\subsection{FPT}
Definir le voisinage, expliquer pourquoi ca nous donne un algo fpt, et le decrire 
\section{Results}
\section{Conclusion}
\end{document}