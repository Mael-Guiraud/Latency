\documentclass{article}
\usepackage[english]{babel}
\usepackage[utf8]{inputenc}
\usepackage{xspace}
\usepackage{graphicx,graphics} 
\usepackage{color}
\usepackage{amsmath}
\usepackage{amsfonts}
\usepackage{amssymb}
\usepackage{amsthm}
\usepackage[utf8]{inputenc}

%opening
\title{Résumé : A fast algorithm for single processor scheduling, Barbara Simons }
\author{Maël Guiraud}

\begin{document}
\maketitle
\parindent=0pt
\section*{Introduction}
L'article ``A fast algorithm for single processor scheduling, de Barbar Simons présente un algorithme efficace pour l’ordonnancement de tâches de même durée
sur un processeur unique. La complexité de cet algorithme est de n²log(n), ou n est le nombre de tâches a ordonnancer.
\section{Modèle}

L'algorithme travail sur un ensemble de tâches qui ont toutes la même durée d’exécution, une release time et une deadline.

Le but est d'ordonnancer les tâches sur un seul processeur de façon à ce que chaque tâche ne débute qu'après son release time et finisse avant sa deadline.

\section{Algorithme}
L'algorithme est composé de trois parties:
 
Une partie naïve, qui choisi la tache prioritaire selon deux critères
\begin{itemize}
 \item On cherche parmi les tâches restantes celles qui sont disponible (c'est à dire dont la release time est inférieure ou égale à la date actuelle).
 \item Parmi ces tâches, on prend la tâche qui à la deadline la plus petite.
\end{itemize}
Lorsqu'une tâche choisie a une deadline inférieure à sa date de départ plus son temps d’exécution, c'est une tâche critique,
on appel la deuxième partie de l'algorithme.

Cette deuxième partie (CRISIS) recherche parmi les tâches déjà ordonnancées une tâche dont la deadline est supérieure à la tache critique.
Si la tache critique est X, la tâche recherchée est appelée PULL(X). Si on ne trouve pas PULL(X), l'algorithme s'arrête.
Cette recherche s'effectue depuis la fin de l'ordonnancement partiel, et la tâche critique ainsi que toutes les taches que l'on à parcouru sont enlevés et placées dans un 
nouvel ensemble appelé ''restricted set``. PULL(X) n'appartiens pas au restricted set.

Une fois ce restricted set établi, on applique l'algorithme naïf pour ordonnancer les taches à l’intérieur. Il peut y avoir une tâche critique,
dans ce cas la, un appel récursif sur CR

Lorsque l'on effectue la recherche sur PULL(X), on ne considère pas les tâches appartenant aux r.s. compris dans le r.s. que l'on est en train 
d'ordonnancer. En revanche, on compte le nombre de tache dans les intervalles entre chaque r.s., et on doit y ordonnancer le même nombre de tâches.
L'intervalle au début du r.s. que l'on est en train d'ordonnancer doit contenir une tache de plus qu'on en a enlevé.

Dans le cas ou l'ordonnancement des tâches dans un intervalle déborde sur le début d'un r.s., on appel la troisième partie de l'algorithme : INVADE.

Cette partie reprends l'ordonnancement du r.s. concerné avec une date de départ qui est celle de la fin de la dernière tâche de l'intervalle qui 
dépassait. Il faut de la même manière réorganiser tous les intervalles du r.s., si un problème cités précédemment se produit, on appel récursivement
la fonction adaptée.


\section{Application}
Dans notre problème, cet algorithme permet d'améliorer l'heuristique ''longest shortest`` proposée. En effet, cette dernière n'utilise que la partie
naïve de l'algorithme pour ordonnancer les messages au retour. Notre heuristique définissant d’abord de façon gloutonne les dates de départ,
on peut calculer le release time de chaque message (symbolisés par des tâches dans l'algorithme), et les deadlines(3ms après l'envoi du message).

Cet algorithme, prouvé comme donnant une solution optimale permet donc d'obtenir la meilleure solution avec les offset de départ choisis.
Il résout donc la moitié du problème en temps polynomial, mais pas tout le problème (appliquer cet algorithme sur chaque configuration de départ 
rends le problème exponentiel).

Cet algorithme ne traite pas de la périodicité du problème, pour l’adapter à notre modèle, il suffit de faire en sorte de définir les deadlines comme
le minimum entre la deadline réelle du message, et la taille de la fenêtre, de façon a pouvoir répéter le processus périodiquement.

\section{Analyse}

La faible complexité de cet algorithme est due au fait que chaque tâche ne pourra être qu'une seule fois critique.
En effet, si une tâche X est une tâche critique, elle va former un r.s. dans lequel chaque tâche aura une deadline inférieure à celle de X.
Donc, si X redeviens critique, PULL(X) n'existera pas et l'algorithme va s'arrêter.

L'algorithme trouve toujours une solution optimal car ....
\end{document}
