\documentclass[10 pt]{beamer}
\usetheme{Madrid}
\usepackage[utf8]{inputenc}
\usepackage{xspace}
\usepackage{graphicx,graphics} 
\usepackage{color}
\usepackage{amsmath}
\usepackage{amsfonts}
\usepackage{amssymb}
\usepackage{amsthm}
\usepackage{tabularx}
\usepackage{algorithm}
\usepackage{algorithmic}
\usepackage{longtable}
\usepackage{complexity}
\usepackage{tkz-graph}
\usepackage{float}
\usepackage{multicol}
\usepackage{setspace}
\usepackage[absolute,overlay]{textpos}
\graphicspath{{fig/}}

\tikzset{
  LabelStyle/.style = { rectangle, rounded corners, draw,
                       font = \bfseries },
  EdgeStyle/.append style = {-} }
\title{ Présentation Maël Guiraud}

\author{{\bf Maël~Guiraud}}


\institute[DAVID-UVSQ] 
{
  DAVID, Universit\'e de Versailles Saint Quentin
   \\
}

\subject{Theoretical Computer Science}

\begin{document}


\begin{frame}

  \titlepage
  \centering
 \includegraphics [width=17.5mm]{logod.png} \\
\end{frame}






\begin{frame}{Profil}
\begin{block}{Formation}
Master 2 AMIS (Algorithmes et Modélisation à l'Interface des Sciences) à l'UVSQ.

Doctorat en informatique de l'université Paris Saclay (thèse CIFRE avec Nokia Bell Labs France).
\end{block}
\begin{exampleblock}{Sujet de thèse}
Ordonnancement periodiques de messages pour minimiser la latence dans les réseaux dans un contexte 5G et au delà.\\
\begin{itemize}
  \item Modélisation précise du problème pratique étudié.
  \item Analyse de compléxité.
  \item Développement de divers algorithmes d'ordonnancement periodiques:
  \begin{itemize}
  \item Algorithmes gloutons avec analyse théorique d'une garantie de résultat.
  \item Algorithmes de résolution exacte des problèmes (FPT).
  \item Utilisation des méta-heuristiques (Descente, recherche tabou, recuit simulé).
  \end{itemize}
  \end{itemize}
\end{exampleblock}

  
\end{frame}

\begin{frame}{Intégration LINEACT}

\begin{block}{Domaines d'applications de mon sujet de thèse:}

 \begin{itemize}
  \item Industrie 4.0
  \item Tour de contrôle à distance
  \item Plus généralement, tout réseau dans lequel on veut limiter la latence au temps physique de transit.
  \end{itemize}
\end{block}
\vspace{1cm}
Je suis actuellement ingénieur de recherche pour le laboratoire HYPHES, une collaboration entre le laboratoire DAVID (Données et Algorithmes pour une Ville Intelligente et Durable) et l'entreprise DCBrain qui travaille sur des sujets appliqués étroitement liés à la ville du futur: 
 \begin{itemize}
  \item Optimisation de reseaux de logistique (camions, taxis, bus...)
  \item Modélisation et optimisation de réseaux urbains et inter-urbains de gaz.
  \end{itemize}


  
\end{frame}

\begin{frame}{Intégration Scientifique}

\textbf{Compétences scientifiques de recherche:}
\begin{itemize}
  \item Modélisation et analyse de la complexité d'un problème.
  \item Plusieurs domaines de l'informatique théorique.
  \item Maitrise des techniques d'optimisation: ordonnancement, méta-heuristiques.
  \item Travaux en cours (labo HYPHES): Techniques de reinforcement learning, théorie des jeux.
  \end{itemize}



  
\end{frame}

\begin{frame}{Enseignements:}
  
    \textbf{Structure de données et Algorithmes}: Licence 2 ($\simeq60$h) et Master 1 ($\simeq30$h)\\
    \textbf{Algorithmes de graph}: Licence 2 ($\simeq30$h)\\
    \textbf{Programmation (C,Java)}: Licence 1-2 (+100h)\\
    \textbf{Sytème d'exploitations}: Pour la formation ISN des professeurs de maths de lycée.\\
    \textbf{Encadrement d'un étudiant de DUT en stage}

\end{frame}





\end{document}
