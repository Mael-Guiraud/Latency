\documentclass[10 pt]{beamer}
\usetheme{Madrid}
\usepackage[utf8]{inputenc}
\usepackage{xspace}
\usepackage{graphicx,graphics} 
\usepackage{color}
\usepackage{amsmath}
\usepackage{amsfonts}
\usepackage{amssymb}
\usepackage{amsthm}
\usepackage{tabularx}
\usepackage{algorithm}
\usepackage{algorithmic}
\usepackage{longtable}
\usepackage{complexity}
\usepackage{tkz-graph}
\usepackage{float}
\usepackage{multicol}
\usepackage{setspace}
\usepackage[absolute,overlay]{textpos}
\graphicspath{{fig/}}

\tikzset{
  LabelStyle/.style = { rectangle, rounded corners, draw,
                       font = \bfseries },
  EdgeStyle/.append style = {-} }
\title{ Deterministic architectures for low latency in 5G and beyond systems}

\author{{\bf Maël~Guiraud}}


\institute[Nokia Bell Labs, DAVID-UVSQ] 
{
  Nokia Bell Labs France - DAVID, Universit\'e de Versailles Saint Quentin
   \\
}

\subject{Theoretical Computer Science}

\begin{document}


\begin{frame}

  \titlepage
  \centering
  \includegraphics [width=25mm]{logon.png} \hspace{1cm} \includegraphics [width=17.5mm]{logod.png} \\
\end{frame}


\begin{frame}

\tableofcontents 
\end{frame}



\section{Introduction (15 min)}
\subsection{Context (5min)}
\begin{frame}{Context}
Presentation communication bout a bout entre deux téléphones, BBU RRH, C-RAN, parler du contexte général 5G et introduire d'autres uses cases potentiels. Insister sur le fait que réduire la latence est important et "à la mode"
\end{frame}
\subsection{Model, problems (10 min)}
\begin{frame}
Présentation comme pour l'exposé Nokia. Model bien clair. Introduction de PALL et SPALL en expliquant bien la différence. Preuves de NP-complétudes
\end{frame}
\section{Solving PALL on simple topologies (15 min)}
\subsection{Star shaped network (2min)}
\begin{frame}
Introduire les star shaped networks, la forme canonique et expliquer qu'on a résolu PAZL dessus
\end{frame}
\subsection{A two stage approach for PALL (8 min)}
\begin{frame}
Expliquer qu'on résout arbitrairement la première étape et qu'on regarde le problème WTA.
Monter creschendo sur WTA en détaillant greedydeadline, MLS et PMLS (FPT aussi ?)
\end{frame}
\subsection{results (5min)}
\begin{frame}{results}
Résultats des algos contre Statistical multiplexing (donc expliquer statistical multiplexing) + avec du traffic Best effort.
\end{frame}

\section{SPALL on random topologies (10 min)}
\subsection{Shape of the topologies(1min)}
\begin{frame}
Tete des instances, pourquoi on peut décortiquer ca en niveaux de contention.
\end{frame}
\subsection{Compact form(3min)}
\begin{frame}
Explication de la forme compacte d'une solution
\end{frame}
\subsection{Local search algorithms + FPT (4min)}
\begin{frame}
Expliquer rapidement qu'on a test descente, tabou et recuit simulé, et passer du temps a expliquer comment est fait le branch and bound.
\end{frame}
\subsection{results(2min)}
\section{Conclusion (5min)}
\begin{frame}{conclusion}
plein de trucs a dire
\end{frame}





\end{document}
