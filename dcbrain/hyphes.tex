
\documentclass{article}
\usepackage[T1]{fontenc}
\usepackage[utf8]{inputenc}

\usepackage[french]{babel}

\usepackage{xspace}
\usepackage{graphicx,graphics} 
\usepackage{color}
\usepackage{amsmath}
\usepackage{amsfonts}
\usepackage{amssymb}
\usepackage{amsthm}
\usepackage{algorithm}
\usepackage{algorithmic}
\usepackage{longtable}
\usepackage{complexity}
\usepackage{hyperref}

\begin{document}



\title{Laboratoire HYPHES}


\maketitle
  
\begin{center}
  \includegraphics [width=25mm]{logodcbrain.png} \hspace{1cm} \includegraphics [width=17.5mm]{logod.png}
\end{center}


\section{Problème 1: Logistique camions simplifié (Pierre)}
  Optimiser le remplissage des camions entre différents entrepôts.

\subsection{Description du problème}
On nous donne un ensemble d'entrepôts (grande distribution type carrefour). Les entrepôts peuvent s'échanger entre eux des colis (marchandises quelconques). Un colis est défini par un entrepôt de départ, un entrepôt d'arrivée et une quantité de marchandise. Les colis peuvent transiter par des entrepôts intermédiaires, par exemple : un camion part de l'entrepôt A avec de la marchandise pour l'entrepôt B et l'entrepôt C. Il dépose toute sa marchandise en B, et un autre camion partira de B vers C avec la marchandise de A et éventuellement la marchandise de B.
On sait que les camions ont une capacité max. Pour abstraire le problème, on considère que les quantités de marchandises n'ont pas de valeurs discrètes, on peut envoyer n'importe quelle décimale de marchandise dans un camion, tant qu'on ne dépasse pas la capacité max. Chaque trajet coute un prix différent. Évidemment plus il y a de camion sur le même trajet, plus le prix est élevé. L'objectif est de trouver quelle quantité de marchandise mettre dans quel camion de façon à minimiser le cout d'envoi des camions.

\subsection{Approche regardée}
Le problème a été modélisé sous forme de programme linéaire. Trop vite incalculable (100 entrepôts, environ 1M de colis par mois et 1K camions).
Pierre propose des approches gloutonnes pour charger les camions.
\subsection{Approche envisagée}
L'objectif au court terme est de proposer un ensemble d'algorithmes gloutons, aidant l'un l'autre à trouver les critères critiques du problème.
À moyen-long terme, il est envisagé d'utiliser les méthodes d'apprentissages par renforcement afin de pouvoir identifier rapidement les caractéristiques critiques et d'utiliser la bonne combinaison d'algorithmes gloutons. Cette approche n'est d'ailleurs pas limitée à ce sujet précisément.

\subsection{Problème plus complexe (Léopold)}
La "vraie" version du problème est en réalité bien plus complexe. Les colis ne peuvent pas être "découpés", le nombre de quais et la place dans les entrepôts ne sont pas illimités, il faut tenir compte des heures d'arrivée (à 5 mins près), et il y a 40 pages de spécifications de ce genre dans le cahier des charges.

\subsection{Problème lié (Maël)}
Maël regarde un problème similaire. Dans un réseau de transport de voitures entre différents concessionnaires, on reçoit régulièrement des nouvelles commandes. L'objectif est de trouver sur quel camion on doit mettre quelle voiture afin d'envoyer le moins de nouveau camions possibles. (à creuser)

\section{Problème 2: Diagramme de Gantt(Max)}
\subsection{Définition du problème}
Alstom sous traite la préparation de chariot de pièces pour construire ses trains. Le client est le sous traitant.
Un bon de commande est envoyé avec la date pour laquelle doit être prêt le chariot. Le chariot doit être prêt pour une date max. On dispose d'une liste d'employés présents, divisés en trois catégories : les animateurs, les packers et les pickers. Les animateurs s'occupent d'organiser les équipes de façon à préparer tous les chariots à temps. Les pickers vont chercher les pièces en entrepôt et les packers les organisent sur le chariot en vérifiant les préparations. 
Pour chaque chariot, on dispose d'une moyenne de temps nécessaire pour le préparer. Certaines pièces nécessitent deux personnes pour être déplacées.
On nous donne donc la liste des chariots à préparer avec leurs "deadlines", et l'effectif disponible.
L'objectif est de trouver la répartition de chaque employé sur chaque chariot à préparer, de façon à optimiser le temps de travail de chacun.

\subsection{Approche regardée}
Max a essayé quelques algorithmes gloutons : placer en premier dans l'emploi du temps des pickers les préparations de chariot les plus longues dès que l'on peut , où les plus courtes, ou placer les préparations aléatoirement. Avec ces approches, il arrive à organiser une bonne quantité des préparations (91 sur 98 sur l'instance qu'il m'a montré), le reste étant à faire à la main.



\section{Remarque globales:}
Après discutions avec les membres de DCbrain, il m'est apparu évident qu'il est très utopique de modéliser un problème finement afin d'obtenir une solution optimale. En effet, des modèles très abstraits seraient vite rattrapés par des contraintes techniques qui rendraient les solutions trouvées mauvaises. De plus, une grande partie les problèmes regardés imposent la contrainte d'un calcul rapide (même si ce n'est pas inscrit dans le cahier des charges).

L'objectif est souvent le même : proposer des bons algorithmes qui trouvent une solution raisonnable rapidement. La plupart des clients contactent DCbrain pour de l'aide à la décision plus que pour la recherche d'optimalité.

%\bibliographystyle{ieeetr}
%\bibliography{Bib}


\end{document}


 
